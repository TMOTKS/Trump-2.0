\documentclass[12pt,a4paper,fontset=fandol]{ctexart}

% 引入必要的宏包
\usepackage[margin=2.5cm]{geometry} % 设置页边距
\usepackage{setspace} % 设置行距
\usepackage{titlesec} % 标题格式设置
\usepackage[colorlinks=true,linkcolor=black,urlcolor=blue,citecolor=black]{hyperref} % 超链接
\usepackage{indentfirst} % 首行缩进

% ==========================================
% 核心修复:将脚注标号修改为 [1], [2] 格式
% ==========================================
\renewcommand{\thefootnote}{[\arabic{footnote}]}

% 格式设置:符合国内学术期刊规范
\onehalfspacing % 1.5倍行距
\titleformat{\section}{\Large\bfseries\centering}{\thesection}{1em}{} % 一级标题居中
\titleformat{\subsection}{\large\bfseries}{\thesubsection}{1em}{}
\renewcommand{\thesection}{\chinese{section}、}
\renewcommand{\thesubsection}{(\chinese{subsection})}

% ==========================================
% 核心修复:去除页眉中的一级标题,保持页面干净
% ==========================================
\pagestyle{plain}

\title{\textbf{科技霸权与认知重塑——特朗普2.0时期对华人工智能战略的逻辑与应对}}
\author{}
\date{}

\begin{document}

\maketitle

\begin{abstract}
\noindent \textbf{【内容摘要】} 人工智能已成为大国战略竞争的核心场域。2026年初,随着特朗普第二任期各项政策的密集落地,美国对华科技战略已从选择性脱钩升级为以“进攻性技术民族主义”为内核的系统性遏制。在这一战略框架下,美国试图通过“星际之门”计划和严苛的出口管制,在物理层实现算力基础设施的排他性构建;同时,利用智能体驱动的认知对抗与算法价值观投射,在认知层谋求话语霸权。然而,该战略的封闭性与零和博弈内核,与科技创新的开放性相冲突,在实践中面临着开源生态冲击、能源瓶颈及盟友战略自主诉求等多重内在悖论与实施困境。面对美国构筑的“数字铁幕”,中国应坚持高水平科技自立自强,构筑开源创新生态,携手“全球南方”践行真正的多边主义,统筹发展与安全,构建国家认知安全与社会韧性体系,以期在全球人工智能治理体系变革中掌握战略主动。

\vspace{1em}
\noindent \textbf{【关键词】} 特朗普2.0;人工智能;进攻性技术民族主义;认知战;大国竞争;数字主权
\end{abstract}

\vspace{2em}

\section*{引言}
\addcontentsline{toc}{section}{引言}

当前,世界正经历百年未有之大变局,大国战略竞争与第四次科技革命相互交织。人工智能技术因其高度的战略性、基础性与军民两用性,已超越单纯的技术属性,成为重塑全球经济、军事与权力结构的关键变量\footnote{鲁传颖、才悦:《特朗普2.0时期中美人工智能博弈的新阶段》,《太平洋学报》2025年第10期,第15页。}。在此背景下,科技创新深度嵌入地缘政治博弈,人工智能安全逐渐成为国家安全体系的核心议题。

2025年至2026年初,随着特朗普再次当选美国总统,美国大战略发生深刻质变。如果说特朗普首个任期的显著特征是粗放式的关税壁垒与实体贸易脱钩,那么其第二个任期则在汲取前期经验的基础上,展现出更为精细且具系统性的重塑特征。特朗普政府将人工智能定位为维系全球科技霸权、控制产业链以及遏制战略竞争对手的关键场域。华盛顿决策层不再满足于维持经济指标的表层领先,而是致力于构建一个由美国主导、排斥战略竞争对手的“技术-认知”闭环体系。

在这一新兴的战略体系中,前沿科技特别是生成式人工智能,被异化为政治权力和地缘博弈的延伸;与此同时,认知领域则被美国军方及情报界定义为继陆、海、空、天、网之后的“第六作战域”,是大国博弈着力争夺的战略高地\footnote{韩娜、邹初妤:《智能认知对抗:理论演化、对抗机制与安全风险》,《国际安全研究》2025年第6期,第49页。}。面对这一历史性的新变局,深入剖析特朗普政府在人工智能基础设施布局与智能认知对抗方面的政策文本,厘清其从物理层面的技术隔离向精神层面的认知霸权升级的战略逻辑,对于准确研判未来国际秩序走向、防范系统性战略风险,以及构建我国的战略应对体系,具有重要的现实意义与理论价值。

\section{特朗普2.0时期人工智能战略的深层动因与逻辑}

特朗普2.0政府在2025年底至2026年初密集出台极具进攻性的科技与认知政策,并非出于偶然的政策冲动,而是基于其核心幕僚团队对国际权力格局演变、国内政治极化态势以及智能化战争形态的综合研判。这种战略转型具有深刻的内生动因与底层逻辑。

\subsection{进攻性技术民族主义与霸权护持}

随着多极化趋势的深入发展,美国战略界对自身综合国力相对衰落的焦虑感明显上升。特朗普2.0政府的经济与安全智囊认为,传统的自由市场机制无法在当前的大国竞争中确保美国在关键技术领域的绝对领先地位。因此,美国政策取向明显向“进攻性技术民族主义”倾斜,试图以非对称的技术代差来重塑全球权力分配。

进攻性技术民族主义是技术政治化进程中的一种理论形态与实践范式,它将技术实力视为国家生存、安全与霸权维系的基础,认为国家最大安全在于垄断能够颠覆现状的“决定性技术优势”\footnote{郭畅、冯晓青:《进攻性技术民族主义与特朗普2.0时期的美国对华人工智能战略》,《太平洋学报》2025年第11期,第59-60页。}。这一逻辑在2026年1月美国总统经济顾问委员会发布的报告《人工智能与大分流》中得到了直白的阐述。该报告提出,美国当前必须利用生成式人工智能的爆发期,引发全球经济与军事实力的第二次“大分流”,以不可逆地拉开与战略竞争对手的差距\footnote{The Council of Economic Advisers, "Artificial Intelligence and the Great Divergence," Executive Office of the President, January 2026, p. 1.}。

为实现这一目标,美国打破了传统的小政府与自由放任理念,转而利用国家意志强行干预技术生态。在《2026国家国防战略》中,美国战争部提出必须进行“如同上个世纪世界大战和冷战期间的国家工业动员”\footnote{Secretary of War, "2026 National Defense Strategy," Department of War, January 2026, p. 21.}。这种基于霸权焦虑的国家级动员逻辑,促使美国试图通过整合全社会的算力、资本与数据,建立起对竞争对手的代差优势,以极化的技术霸权来护持其单极霸权体系。

\subsection{“科技-工业复合体”的利益驱动与“内政外化”}

特朗普2.0战略的另一显著特征,是国内政治生态对外交与科技政策的深刻形塑。近年来,美国国内政治愈发成为影响其外交政策的关键变量,“内政驱动外交”的态势日益凸显。特朗普政府通过将国内治理困境归咎于外部威胁,呈现出明显的“内政外化”治理逻辑\footnote{孙涵:《特朗普“美国优先2.0”外交战略及其内政驱动因素》,《国际论坛》2025年第5期,第25页。}。

在这一逻辑下,硅谷科技精英与右翼政治力量的合流发挥了关键作用。部分科技精英推崇“技术加速主义”,认为技术的自驱进步将促进社会发展,主张探索能够在释放技术红利与赚取商业利益间保持平衡的路径\footnote{金灿荣、申欣钰:《特朗普政府人工智能政策的国家安全逻辑》,《太平洋学报》2025年第8期,第45页。}。特朗普与这部分科技精英达成共识,形成了一种“以权力换利润、以利润换权力”的相互支持关系\footnote{马贺非、毛维准:《特朗普政府人工智能基础设施政策探析》,《现代国际关系》2025年第4期,第13-14页。}。科技巨头通过提供私人资本换取在政策制定中的话语权,而政府则通过支持科技公司的创新发展来兑现其政治承诺。这种“科技-工业复合体”的深度内嵌,不仅推动了美国人工智能政策向重创新、轻监管的方向倾斜,也使得技术议题不可避免地被意识形态化和安全化,成为转移国内矛盾、凝聚选民基础的政治工具。

\subsection{智能化战争形态演进与本体安全焦虑}

生成式人工智能与智能体技术的突破性进展,为特朗普政府重塑军事与认知战略提供了现实驱动力。当前,人工智能技术正处于从专用智能向通用人工智能跃迁的临界点。美国战争部深刻认识到,技术赋能的战争模式和武器能力开发,将在未来十年重新定义军事事务的特征与战争的制胜机理\footnote{Secretary of War, "Memorandum: Artificial Intelligence Strategy for the Department of War," Washington, D.C., January 9, 2026, p. 1.}。

与此同时,美国面临着传统物质安全与国家“本体安全”的双重压力。本体安全指国家内部免受社会分裂、认同销蚀、政府合法性削弱等威胁的状态与能力\footnote{金灿荣、申欣钰:《特朗普政府人工智能政策的国家安全逻辑》,《太平洋学报》2025年第8期,第44页。}。为了应对这些挑战,特朗普政府确立了将美军全面转型为“人工智能优先”作战力量的宏大目标。这一转型的核心逻辑在于“速度制胜”,即必须将学习速度武器化,把作战循环时间和技术采用率作为决定性的战场变量,以超越潜在对手的适应与反应能力\footnote{Secretary of War, "Memorandum: Artificial Intelligence Strategy for the Department of War," Washington, D.C., January 9, 2026, p. 4.}。这种对智能化战争形态的追求和对绝对军事优势的执念,直接催生了特朗普政府在数据整合、算力垄断和心理战升级等领域的一系列政策举措。
\section{物理层锁定:AI基础设施的战略动员与排他性构建}

在进攻性技术民族主义的驱动下,特朗普2.0政府在硬实力层面构建了一套严密的物理层锁定机制。这套机制从算力基建、数据要素、采办制度到国际供应链,形成了一个相对封闭的体系,意在从源头控制未来前沿技术的发展路径。

\subsection{算力垄断与“星际之门”计划的全政府动员}

在人工智能时代,算力与高质量数据是定义未来技术权力的核心生产资料。特朗普政府试图通过行政资源引导与公私合作,将分散的算力资源进行国家级整合。

2025年1月,特朗普宣布启动“星际之门”计划,预计投资5000亿美元用于建设人工智能基础设施,特别是超大规模数据中心\footnote{郑执浩:《美国人工智能建设的新发展:“星际之门”计划前景分析》,《现代国际关系》2025年第4期,第25-26页。}。该计划并非单纯的商业投资,而是具有明显的战略动员色彩。它采取公私合作模式,由私营部门提供资金并主导运营,而联邦政府则通过土地使用、税收减免和简化审批流程等方式提供政策便利。这种紧密的“政府主导-科技资本”联系,体现了美国试图通过大规模基础设施建设,确立其在人工智能领域的绝对优势。

为配合算力层面的整合,美国军方在数据基座的重组上也采取了相应举措。2026年初,美国战争部宣布将原有的数据平台进行重组,集中打造“战争数据平台”。这一举措旨在打破军种间的数据壁垒,构建一个能支撑智能体在全军范围内快速开发与集成的底层数据枢纽\footnote{Deputy Secretary of War, "Memorandum: Transforming Advana to Accelerate Artificial Intelligence and Enhance Auditability," Washington, D.C., January 9, 2026, pp. 1-2.}。通过数据要素的集中管理和算力资源的大规模投入,美国政府试图在科学发现与战争决策的底层逻辑上建立起对竞争对手的技术代差。

\subsection{技术标准的“去伦理化”与敏捷响应机制}

在传统的大国科技竞争中,繁琐的官僚体制往往会延缓新技术向实际能力的转化。为支撑高强度的科技竞赛,特朗普政府大力推动体制改革,试图赋予庞大的政府和军事机器以更高的敏捷性。

在政策导向上,特朗普政府推行“创新优先、监管最小化”的逻辑。2025年1月签署的第14179号行政令,废除了拜登政府时期关于安全、可靠地开发和使用人工智能的行政令,旨在消除被其视为阻碍创新的监管障碍\footnote{马贺非、毛维准:《特朗普政府人工智能基础设施政策探析》,《现代国际关系》2025年第4期,第8-9页。}。这种“去伦理化”的政策倾向,反映了其将竞争优势置于安全考量之上的实用主义考量。

在军事采办层面,美国战争部于2026年初完成了国防创新生态的重组,确立了负责研究与工程的副部长作为唯一的“首席技术官”,统摄全局技术方向。同时,设立“障碍移除委员会”以豁免繁琐的审批程序,并依托《大美法案》提供的资金,设立“创新插入增量”预算机制\footnote{Secretary of War, "Memorandum: Transforming the Defense Innovation Ecosystem to Accelerate Warfighting Advantage," Washington, D.C., January 9, 2026, pp. 1-4.}。这一系列制度设计的明确目标是:在商业前沿模型公开发布的短期内,完成其在美军作战系统中的部署。这折射出特朗普政府试图通过制度创新,将商业技术优势快速转化为即时安全优势的战略企图。

\subsection{“硅基和平”与全球产供应链的阵营化重塑}

在确保自身技术研发与基础设施建设领先的同时,特朗普政府在国际舞台上推出了具有较强排他性的技术输出与供应链重塑战略。

美国试图通过出口管制等强制性技术权力,限制竞争对手获取先进技术资源。2025年,美国商务部工业与安全局发布新规,对高端人工智能芯片的管制进一步升级,并引入“三级穿透信息”披露的审查方式,追溯人工智能技术的军民两用风险\footnote{郭畅、冯晓青:《进攻性技术民族主义与特朗普2.0时期的美国对华人工智能战略》,《太平洋学报》2025年第11期,第65-66页。}。此外,美国还通过“芯片四方联盟”等机制,施压盟友在技术投资、标准制定上配合其出口管制政策。

另一方面,美国推行《促进美国人工智能全栈技术出口计划》,向全球盟友及部分发展中国家推销包含优化硬件、数据通道、云服务及底层大模型在内的整体解决方案。美国总统经济顾问委员会在报告中提及了名为“硅基和平”的国际供应链联盟,试图将处于供应链不同环节的国家捆绑在美国的技术架构内\footnote{The Council of Economic Advisers, "Artificial Intelligence and the Great Divergence," Executive Office of the President, January 2026, p. 6.}。这种策略旨在通过物理层锁定和标准捆绑,促使其他国家在数字基础设施上采纳美国架构,从而在全球范围内重塑符合美国利益的人工智能价值链。

\section{认知域重塑:智能体驱动的认知对抗与话语霸权}

进入特朗普2.0时期,美国政府的科技战略已超越了硬实力竞争的传统范畴,转而将人工智能特别是大语言模型的话语生成能力,作为重塑全球认知结构、界定国际规则的工具。认知战被进一步系统化,成为其大国竞争战略的重要组成部分。

\subsection{智能体驱动的认知对抗机制升级}

美国军事战略文化正在经历从隐蔽的信息干预向更具系统性的认知操纵转型。2025年底,美国战争部发布备忘录,恢复使用“心理战”这一术语,明确其旨在影响外国政府、组织、群体及个人的情感、动机、客观推理,并最终改变其行为\footnote{Secretary of War, "Memorandum: Changing the Term Military Information Support Operations Back to Psychological Operations," Washington, D.C., December 2, 2025.}。

这一战略回归在《战争部人工智能战略》中得到了技术支撑。随着智能体技术的应用,认知对抗的模式发生了显著升级。智能体具备自主感知环境、制定策略并采取行动的能力,使得认知干预从传统的单向信息传播,演变为“感知-决策-行动”闭环的动态交互过程\footnote{韩娜、邹初妤:《智能认知对抗:理论演化、对抗机制与安全风险》,《国际安全研究》2025年第6期,第52-53页。}。例如,美军启动的“蜂群工场”和“智能体网络”等项目,旨在利用智能体直接参与从战役规划到杀伤链执行的决策闭环。这意味着美军正试图将操纵对手认知、影响决策意志视为与物理火力打击同等重要的作战能力。在智能体的加持下,认知对抗的规模、速度和隐蔽性都将大幅提升。

\subsection{算法价值观投射与制度性权力扩张}

在数字时代,大语言模型的伦理标准和底层算法逻辑直接影响着信息的筛选与历史叙事的构建。特朗普政府意识到,人工智能模型并非价值中立的工具,而是蕴含着意识形态能量的载体。

为了确保美国对外输出的认知叙事符合其战略利益,特朗普政府采取了行政干预手段。2025年12月签署的第14365号行政令《确保国家政策框架》中,特朗普政府动用联邦优先权,授权成立诉讼特别工作组,以应对部分州政府试图立法禁止算法歧视的举动\footnote{Executive Order 14365, "Ensuring a National Policy Framework for Artificial Intelligence," Federal Register, Vol. 90, No. 239, December 16, 2025, pp. 58499-58500.}。

结合战争部对采购中“任何合法用途”的要求,美国试图通过制度性权力,清除模型中不符合其保守主义价值观的参数,确立一套具有强烈“美国优先”色彩的统一算法标准。当这套经过特定价值观塑造的算法标准,随着全栈出口计划推向全球时,实际上是在向接收国投射其话语霸权。数字巨头作为技术权力的实际行使者,通过控制知识的生产、传播以及分配规则,影响其他行为体的认知框架与决策空间\footnote{孙志伟、殷浩铖:《人工智能时代数字巨头的技术权力及其对“全球南方”的挑战》,《国际安全研究》2025年第2期,第148-149页。}。这将在潜移默化中重塑全球特别是发展中国家民众的历史观与价值观。

\subsection{信息污染与全球认知主权的系统性侵蚀}

生成式人工智能的高效信息生成能力,为认知战提供了低成本制造海量信息的技术手段。这不仅提升了认知战的效能,也带来了信息污染和认知主权侵蚀的风险。

在认知战中,进攻方可以使用生成式人工智能工具制造大量虚假信息或具有特定倾向性的误导信息。这些信息在社交媒体等网络空间中快速传播,容易形成“信息茧房”和“回声室效应”,阻碍受众获取真实、全面的信息\footnote{贾子方、王栋:《生成式人工智能对国际安全的影响:以认知战为路径的分析》,《国际政治研究》2025年第3期,第91-92页。}。长期的信息污染不仅会扭曲个体的认知判断,还可能加剧社会群体的认知分裂和对立情绪,进而引发社会信任危机和政治动荡。

此外,美国通过将国家安全观念的灌输与技术标准的同化前推至基础教育阶段,试图从代际源头巩固其认知优势。第14277号行政令《推进美国青少年的教育》提出建立白宫教育特别工作组,将国家安全需求与基础教育深度融合\footnote{Executive Order 14277, "Advancing Artificial Intelligence Education for American Youth," Federal Register, Vol. 90, No. 80, April 28, 2025, pp. 17519-17520.}。这种战略布局旨在培养对美式技术标准有天然亲和力的后备力量,确保美国在长期竞争中拥有稳固的人力资源储备。这种从技术底层到教育源头的全方位认知塑造,对其他国家的认知主权和文化安全构成了系统性挑战。
\section{特朗普2.0时期人工智能战略的内在悖论与实施困境}

特朗普2.0政府通过“星际之门”计划重塑算力基建,以及将认知对抗系统化等一系列举措,试图构建起一张精密的技术与认知封锁网。然而,这一基于进攻性技术民族主义的战略,其排他性、封闭性与零和博弈内核,与全球化时代科技创新的开放性、协同性相冲突。在实践中,该战略不可避免地陷入系统性困境,面临着多重内在悖论与外部挑战。

\subsection{封闭垄断与开源创新的结构性冲突}

进攻性技术民族主义驱使美国采用技术封锁与规则排他等手段压制他国创新。然而,这一逻辑与现代技术创新规律存在深刻矛盾。人工智能的演进高度依赖全球性创新网络、海量数据与多样化场景,封闭体系最终可能拖慢其自身技术迭代速度。

中国在人工智能领域的非对称创新突破,凸显了美国封锁战略的局限性。面对美国的算力与芯片管制,中国企业并未完全陷入被动,而是通过算法优化、场景创新与开源生态建设,探索出一条不同的技术路径。例如,中国企业推出的DeepSeek-R1等开源模型,以较低的算力投入,在多项关键基准测试中展现出比肩甚至超越西方前沿模型的性能\footnote{鲁传颖、才悦:《特朗普2.0时期中美人工智能博弈的新阶段》,《太平洋学报》2025年第10期,第21页。}。这种“算力约束下的模型优化”路径,证明了开源模式在汇聚全球智慧、降低创新门槛方面的巨大潜力。

中国开源模型的崛起,不仅打破了美国科技巨头对人工智能模型的垄断,也对美国试图构建的封闭技术生态形成了有力冲击。这表明,单纯的技术封锁难以遏制具有完备工业体系和庞大人才基数的大国崛起,反而可能激发其自主创新的内生动力,加速全球技术体系的多极化发展。

\subsection{算力扩张的能源瓶颈与复合体的自限性}

美国试图通过“星际之门”等超大规模基础设施项目确立绝对的算力优势,但这一宏大计划在实施过程中面临着严峻的现实制约。

首先是能源供给的瓶颈。生成式人工智能数据中心是名副其实的“耗电巨兽”。据预测,到2030年,美国数据中心的电力需求将大幅增长,占全国总电力需求的比例将显著上升\footnote{马贺非、毛维准:《特朗普政府人工智能基础设施政策探析》,《现代国际关系》2025年第4期,第19-20页。}。然而,美国老化的电网设施、漫长的能源项目审批流程以及对传统化石能源的依赖,使得电力供应的稳定性和可持续性面临巨大挑战。能源缺口可能成为制约美国人工智能基础设施扩张的“阿喀琉斯之踵”。

其次是“科技-工业复合体”内部的利益分歧与自限性。美国人工智能基础设施的建设高度依赖少数科技巨头。虽然这些企业在推动技术创新方面发挥了重要作用,但其逐利本性也导致了复合体内部的激烈竞争与利益冲突。例如,在“星际之门”项目中,参与方在技术路线、资本合作及数据中心运营等方面存在潜在的分歧\footnote{郑执浩:《美国人工智能建设的新发展:“星际之门”计划前景分析》,《现代国际关系》2025年第4期,第28-29页。}。此外,过度依赖少数企业可能导致技术发展路径的僵化,削弱国家整体的创新韧性。

\subsection{盟友战略自主诉求与“数字铁幕”的推行阻力}

美国试图通过联盟体系和“硅基和平”等排他性机制,在全球范围内推行其技术标准和出口管制政策。然而,这一战略在实施中遭遇了来自盟友及全球南方国家的普遍抵制。

一方面,美国将联盟工具化、交易化的做法,忽视了盟友的经济利益与战略自主性。美国施压盟友对华实施先进技术禁售,直接冲击了相关国家企业的市场份额,引发了盟友内部的不满。在经济利益和战略自主的考量下,欧盟、日本等关键行为体在对华技术政策上与美国产生分歧,纷纷推进产业自主计划,以降低对美技术依赖\footnote{郭畅、冯晓青:《进攻性技术民族主义与特朗普2.0时期的美国对华人工智能战略》,《太平洋学报》2025年第11期,第70页。}。

另一方面,广大全球南方国家对美国试图构建的“数字铁幕”保持高度警惕。发展中国家渴望搭乘人工智能技术发展的快车,但不希望沦为大国竞争的棋子或数字附庸。美国附加政治条件的技术输出,引发了这些国家对数字主权丧失的深层担忧。因此,美国难以构建起铁板一块的遏华技术联盟,其试图垄断全球人工智能价值链的战略目标面临重重阻力。

\section{跨越“数字铁幕”:中国的战略应对与多边主义实践}

面对特朗普2.0政府在人工智能领域构筑的技术与认知防线,中国需准确研判其战略意图。在保持战略定力的同时,应统筹发展与安全,采取灵活务实的应对之策,以真正的多边主义和高水平科技自立自强,化解外部战略挤压。

\subsection{坚持高水平科技自立自强,构筑开源创新生态}

针对美国的技术封锁与“敏捷突袭”风险,中国必须将科技自立自强作为国家发展的战略支撑。应充分发挥新型举国体制优势,加强原始创新和关键核心技术攻关,在人工智能芯片、关键算法等底层技术上实现非对称突破。

同时,应大力倡导和构筑开源创新生态。开源模式不仅能够汇聚全球开发者的智慧,加速技术迭代,还能有效规避单一边界的技术封锁。中国应积极参与并主导国际开源社区的建设,支持国内优秀的人工智能模型开源开放,为全球特别是广大发展中国家提供低成本、高性能的技术选项。通过构建“底层技术-应用场景-资本支持”的完整生态,形成具有国际竞争力的自主技术体系,从根本上打破美国的技术垄断。

\subsection{携手“全球南方”践行真正的多边主义}

美国推行的排他性技术输出,附带极强的政治条件,这恰恰触及了全球南方国家的战略痛点。中国应抓住这一契机,全面升级技术出海策略,从单纯的产品输出转向全方位的“技术赋能”。

依托《全球人工智能治理倡议》和全球发展倡议,中国应高举“数字主权”和“技术不结盟”的旗帜,向全球南方国家提供去捆绑、开源可控的替代性人工智能与算力基础设施方案\footnote{李猛:《中国携手“全球南方”倡导践行真正的多边主义的实践与展望》,《南亚研究》2025年第3期,第21页。}。通过帮助发展中国家建立本土化的数据中心、培养本土技术人才,中国可以助力其跨越“数字鸿沟”。在联合国等多边框架下,中国应与全球南方国家一道,共同参与人工智能国际规则的制定,推动全球人工智能治理朝着更加公平、包容、普惠的方向发展,构建广泛的国际数字统一战线。

\subsection{统筹发展与安全,构建国家认知安全与社会韧性体系}

面对智能化认知对抗带来的政治安全、社会极化与文化侵蚀风险,中国必须从总体国家安全观的高度,构建全面、动态的认知安全治理体系。

在技术层面,应超前布局认知安全关键技术,建立从感知、阻断到主动防御的智能化防御链。强化对深度伪造、自动化虚假信息生成等认知攻击手段的识别与反制能力。在制度层面,应完善人工智能相关法律法规,明确平台与开发者的责任边界,设立认知风险影响评估制度。在社会层面,应系统性构建社会认知韧性。通过国民教育强化全民特别是青少年的信息素养与批判性思维,提升公众在复杂信息环境下的辨别力与“认知免疫力”\footnote{韩娜、邹初妤:《智能认知对抗:理论演化、对抗机制与安全风险》,《国际安全研究》2025年第6期,第67页。}。同时,加快培养兼具前沿技术背景与国际政治视野的复合型战略人才,为应对未来的认知域对抗提供坚实的人才保障。

\section*{结语}
\addcontentsline{toc}{section}{结语}

特朗普2.0政府在人工智能领域的战略转型,是一次试图通过行政手段和国家资本力量,强行逆转多极化历史周期的大国博弈。其核心逻辑在于透支现存的技术存量与金融霸权,通过扭曲市场规则、重塑科学范式与输出特定价值观,构建一个排他性的“科技-认知”闭环体系,以期维持其全球霸权地位。

然而,从长远的历史视角审视,这种基于进攻性技术民族主义的战略不可避免地带有内在的脆弱性与逻辑悖论。试图通过行政权力垄断技术发展路径,背离了科技创新依赖开放与共享的客观规律;将算法标准异化为霸权工具,难以获得国际社会的广泛认同,反而会激发全球范围内的防御性反弹。

对于中国而言,面对外部的“数字铁幕”与认知攻势,应始终保持高度的战略定力与制度自信。历史经验表明,单纯的技术封锁难以遏制一个拥有完备工业体系和庞大创新潜力的大国的崛起。只要中国坚持高水平科技自立自强,坚定维护国家数字主权,积极践行真正的多边主义,持续拓展平等互利、包容普惠的国际技术合作网络,就一定能在百年未有之大变局中掌握战略主动,为推动构建人类命运共同体贡献中国智慧与中国力量。

\end{document}
