\documentclass[12pt,a4paper,fontset=fandol]{ctexart}

% 引入必要的宏包
\usepackage[margin=2.5cm]{geometry} % 设置页边距
\usepackage{setspace} % 设置行距
\usepackage{titlesec} % 标题格式设置
\usepackage[colorlinks=true,linkcolor=black,urlcolor=blue,citecolor=black]{hyperref} % 超链接
\usepackage{indentfirst} % 首行缩进

% ==========================================
% 核心修复:将脚注标号修改为 [1], [2] 格式,符合国内顶刊规范
% ==========================================
\renewcommand{\thefootnote}{[\arabic{footnote}]}

% 格式设置:符合国内学术期刊规范
\onehalfspacing % 1.5倍行距
\titleformat{\section}{\Large\bfseries\centering}{\thesection}{1em}{} % 一级标题居中
\titleformat{\subsection}{\large\bfseries}{\thesubsection}{1em}{}
\renewcommand{\thesection}{\chinese{section}、}
\renewcommand{\thesubsection}{(\chinese{subsection})}

% 去除页眉,保持页面干净严肃
\pagestyle{plain}

\title{\textbf{科技霸权与认知重塑——特朗普2.0时期对华人工智能战略的逻辑与应对}}
\author{}
\date{}

\begin{document}

\maketitle

\begin{abstract}
\noindent \textbf{【内容摘要】} 在第四次工业革命与大国战略竞争深度交织的时代背景下,人工智能已成为重塑国际权力结构与全球安全格局的核心场域。2026年初,随着特朗普第二任期各项政策的密集落地,美国对华科技战略已从首个任期的选择性脱钩与经贸保护主义,深刻演进为以“进攻性技术民族主义”为内核的系统性遏制与霸权护持战略。在《2025国家安全战略》和《2026国家国防战略》的宏观指导下,特朗普政府通过启动“创世纪任务”、实施“战争部全面转型为人工智能优先作战力量”及重塑国防创新生态,构建了一套集物理控制与认知塑造于一体的战略体系。该体系在硬实力层面,试图通过国家意志强力调配算力与数据资源,建立垄断性的科研与采办体系,以期引发全球经济与军事实力的第二次“大分流”;在软实力层面,将心理战制度化、常态化,并试图通过“反觉醒”技术标准的输出确立排他性的认知霸权。本文基于2025年至2026年初美国联邦政府发布的关键行政令和政策档案,系统梳理特朗普2.0政府科技与认知新政的实施路径,深度剖析其背后的权力运作逻辑、内在悖论,并提出我国在跨越“数字铁幕”、构建全球人工智能治理新秩序中的战略应对路径。

\vspace{1em}
\noindent \textbf{【关键词】} 特朗普2.0;人工智能;进攻性技术民族主义;认知战;大国竞争;数字主权
\end{abstract}

\vspace{2em}

\section*{引言}
\addcontentsline{toc}{section}{引言}

当前,世界正经历百年未有之大变局,国际政治经济格局处于冷战后最为剧烈的震荡与重组期。这一历史性变局的核心特征之一,是大国战略竞争与第四次科技革命的同频共振。在这一进程中,人工智能技术因其高度的战略性、基础性、渗透性与军民两用性,已彻底超越了单纯的技术工具属性,成为塑造国际体系权力分配、重构地缘政治秩序以及决定国家兴衰的关键战略变量\footnote{鲁传颖、才悦:《特朗普2.0时期中美人工智能博弈的新阶段》,《太平洋学报》2025年第10期,第15-16页。}。科技创新不仅深度嵌入地缘政治博弈的各个环节,人工智能安全更已跃升为国家总体安全体系中不可或缺的核心议题。

2025年至2026年初,随着特朗普再次当选并正式就任美国总统,美国的大战略与对外政策发生了深刻质变。如果说特朗普首个任期的显著特征是粗放式的关税壁垒、退群毁约与实体贸易的硬性脱钩,那么其第二个任期则在汲取前期经验与整合国内政治力量的基础上,展现出更为精细、冷酷且具高度系统性的重塑特征。特朗普政府明确将人工智能定位为维系全球科技霸权、控制未来产业链以及全面遏制战略竞争对手的关键场域。华盛顿的决策层已不再满足于维持国内生产总值增速或制造业回流等传统经济指标的表层领先,而是致力于构建一个由美国绝对主导、严格排斥战略竞争对手的“技术-认知”闭环体系。

在这一新兴的战略体系中,前沿科技特别是生成式人工智能、量子计算与先进半导体,被彻底剥离了其作为人类公共产品的客观属性,被异化为政治权力和地缘博弈的直接延伸。与此同时,随着信息技术的迭代与社交媒体的普及,认知领域则被美国军方及情报界明确定义为继陆、海、空、天、网之后的“第六作战域”,成为大国博弈着力争夺的战略高地\footnote{韩娜、邹初妤:《智能认知对抗:理论演化、对抗机制与安全风险》,《国际安全研究》2025年第6期,第48-49页。}。美国试图通过算法、算力与数据的深度融合,实现从传统的“信息控制”向深层的“认知塑造”与“意识形态同化”转变。

面对这一历史性的新变局,系统梳理并深入剖析特朗普政府在人工智能基础设施布局与智能认知对抗方面的政策文本,厘清其从物理层面的技术隔离向精神层面的认知霸权升级的战略逻辑,显得尤为迫切。这不仅有助于准确研判未来国际秩序的演变走向、防范系统性与结构性的战略风险,更为我国在复杂多变的国际环境中保持战略定力、构建自主可控的技术体系与国家认知安全防线,提供了重要的现实依据与理论支撑。本文将以美国联邦政府及相关部门在2025年至2026年初发布的最新政策档案为基础,从战略动因、物理层锁定、认知域重塑、内在悖论及中国应对五个维度,对特朗普2.0时期的科技与认知战略展开全面透视。

\section{特朗普2.0时期人工智能战略的深层动因与逻辑}

特朗普2.0政府在2025年底至2026年初密集且激进地出台了一系列极具进攻性的科技与认知政策。这些政策的集中爆发并非出于偶然的政策冲动或单纯的选举承诺兑现,而是基于其核心幕僚团队对国际权力格局演变趋势、国内政治极化态势以及智能化战争形态的综合研判。这种被美国战略界内部称为“新曼哈顿”式的大规模战略转型,具有深刻的内生动因与底层逻辑。

\subsection{进攻性技术民族主义与谋求第二次“大分流”的霸权护持}

随着世界多极化趋势的深入发展和以金砖国家为代表的“全球南方”群体性崛起,美国战略界对自身综合国力相对衰落的焦虑感明显上升。特朗普2.0政府的经济与安全智囊们深刻认识到,冷战后确立的传统自由市场机制和新自由主义全球化,不仅导致了美国本土产业的空心化,更无法在当前激烈的大国竞争中确保美国在关键技术领域的绝对领先地位。因此,美国的科技政策取向发生了根本性逆转,明显向“进攻性技术民族主义”倾斜,试图以非对称的技术代差来强行重塑全球权力分配格局。

进攻性技术民族主义是技术政治化进程中的一种极端理论形态与实践范式。它将技术实力视为国家生存、安全与霸权维系的绝对基础,认为在无政府状态的国际体系中,国家的最大安全在于垄断能够颠覆现状的“决定性技术优势”\footnote{郭畅、冯晓青:《进攻性技术民族主义与特朗普2.0时期的美国对华人工智能战略》,《太平洋学报》2025年第11期,第59-60页。}。与防御性技术民族主义旨在实现技术追赶不同,进攻性技术民族主义具有鲜明的压制性、排他性与零和性,其核心目标是通过系统性的政治、经济与制度手段,对特定竞争对手实施技术压制与体系排斥。

这一底层逻辑在2026年1月美国总统经济顾问委员会发布的重磅报告《人工智能与大分流》中得到了最为直白和露骨的阐述。该报告明确提出,正如18至19世纪的工业革命导致了西方与非西方世界在财富与权力上的第一次“大分流”,美国当前必须紧紧抓住生成式人工智能爆发的历史机遇,利用这一通用目的技术引发全球经济与军事实力的第二次“大分流”,以彻底且不可逆地拉开与战略竞争对手的差距\footnote{The Council of Economic Advisers, "Artificial Intelligence and the Great Divergence," Executive Office of the President, January 2026, p. 1.}。

为实现这一宏大且充满野心的目标,美国打破了共和党传统的小政府与自由放任理念,转而利用国家意志强行干预技术生态。在2026年1月发布的《国家国防战略》中,美国战争部直言不讳地提出,面对日益严峻的国际竞争,必须进行“如同上个世纪世界大战和冷战期间的国家工业动员”\footnote{Secretary of War, "2026 National Defense Strategy," Department of War, January 23, 2026, p. 21.}。这种基于深层霸权焦虑的国家级动员逻辑,促使美国试图通过整合全社会的算力、资本与数据资源,建立起对竞争对手的代差优势。其核心目的在于,面对《2026国家国防战略》中重点担忧的多战区“同时性问题”\footnote{Ibid., p. 13.},通过技术垄断来弥补美军在常规兵力规模上的相对不足,以极化的技术霸权来强行护持其摇摇欲坠的单极霸权体系。

\subsection{“科技-工业复合体”的利益驱动与“内政外化”的政治逻辑}

特朗普2.0科技战略的另一显著特征,是国内政治生态对外交与科技政策的深刻形塑。近年来,美国国内政治愈发成为影响其外交政策的关键变量,“内政驱动外交”的态势日益凸显。特朗普政府通过娴熟运用“外部威胁”的政治叙事来解释国内的结构性矛盾,呈现出明显的“内政外化”治理逻辑\footnote{孙涵:《特朗普“美国优先2.0”外交战略及其内政驱动因素》,《国际论坛》2025年第5期,第25-26页。}。

在这一逻辑下,硅谷科技精英与右翼保守主义政治力量的深度合流发挥了关键的驱动作用。部分科技精英推崇“技术加速主义”,认为技术的自驱进步将促进社会发展与国家实力的跃升,主张探索能够在释放技术红利与赚取商业利益间保持平衡的路径\footnote{金灿荣、申欣钰:《特朗普政府人工智能政策的国家安全逻辑》,《太平洋学报》2025年第8期,第45-46页。}。特朗普在竞选期间及上任后,与这部分科技精英达成了高度共识,形成了一种“以权力换利润、以利润换权力”的相互支持关系\footnote{马贺非、毛维准:《特朗普政府人工智能基础设施政策探析》,《现代国际关系》2025年第4期,第13-14页。}。科技巨头通过提供庞大的私人资本和技术支持,换取在联邦政策制定、监管豁免以及政府巨额采购订单中的话语权;而政府则通过支持科技公司的创新发展,来兑现其关于经济复苏和重振美国领导力的政治承诺。

这种“科技-工业复合体”的深度内嵌,不仅推动了美国人工智能政策向重创新、轻监管的方向大幅倾斜,也使得技术议题不可避免地被高度意识形态化和安全化。特朗普政府将国内的右翼保守主义与“身份政治”诉求,强行延伸至国际战略和技术标准领域。例如,在《2025国家安全战略》中,美国政府明确将“能力与优绩主义”列为国家复兴的核心原则,并以前所未有的严厉措辞,公开反对“多元、公平、包容”等被其视为“削弱美国竞争力”、“破坏制度”的激进意识形态\footnote{President of the United States, "National Security Strategy of the United States of America," The White House, November 2025, p. 11.}。这种将技术标准与国内文化战争深度绑定的做法,本质上是将科技政策作为转移国内阶级矛盾、凝聚基本盘选民的政治工具。

\subsection{智能化战争形态演进与国家本体安全诉求}

除了宏观的政治与经济因素,生成式人工智能与智能体技术的突破性进展,为特朗普政府重塑军事与认知战略提供了最直接的现实物理驱动力。当前,人工智能技术正处于从专用智能向通用人工智能跃迁的临界点。美国战争部深刻认识到,技术赋能的战争模式和武器能力开发,将在未来十年彻底重新定义军事事务的特征与战争的制胜机理\footnote{Secretary of War, "Memorandum: Artificial Intelligence Strategy for the Department of War," Washington, D.C., January 9, 2026, p. 1.}。

与此同时,美国不仅面临着传统物质安全(如地缘政治、军事威慑)的压力,更面临着深层的国家“本体安全”焦虑。本体安全是指国家内部免受社会分裂、认同销蚀、政府合法性削弱等威胁的状态与能力,体现为国内社会的凝聚与稳定\footnote{金灿荣、申欣钰:《特朗普政府人工智能政策的国家安全逻辑》,《太平洋学报》2025年第8期,第44页。}。面对国内政治极化、信息茧房加剧以及外部认知干预的担忧,美国政府急于通过掌控最先进的人工智能技术,来重塑国内信息环境的控制力,从而维护其本体安全。

为了应对这些内外双重挑战,特朗普政府确立了将美军全面转型为“人工智能优先”作战力量的宏大目标。根据2026年1月发布的《战争部AI战略》备忘录,这一转型的核心逻辑可以概括为“速度制胜”。该战略明确指出,在可预见的未来,军事较量是一场纯粹的竞速赛,美国必须将“学习速度武器化”,把作战循环时间和技术采用率作为决定性的战场变量,以超越任何潜在对手的适应与反应能力\footnote{Secretary of War, "Memorandum: Artificial Intelligence Strategy for the Department of War," Washington, D.C., January 9, 2026, p. 4.}。这种对智能化战争形态的狂热追求和对绝对军事优势的执念,直接催生了特朗普政府在数据整合、算力垄断和心理战升级等领域的一系列激进且具有颠覆性的政策举措。
\section{物理层锁定:人工智能基础设施的战略动员与排他性构建}

在进攻性技术民族主义的强烈驱动下,特朗普2.0政府在硬实力层面彻底抛弃了自由市场在全球资源配置中的主导地位,转而构建了一套严密且极具排他性的“物理层锁定”机制。这套机制从算力基础设施建设、数据要素垄断、军事采办制度改革到国际供应链重塑,形成了一个高度闭环的系统,意在从源头控制未来前沿技术的发展路径,剥夺竞争对手的追赶空间。

\subsection{算力垄断与“星际之门”计划的全政府动员}

在人工智能时代,庞大的算力集群与高质量的大规模数据是定义未来技术权力的核心生产资料。特朗普政府试图通过行政资源引导与公私合作模式,将分散的算力资源进行国家级的整合与垄断。

2025年1月,特朗普在就职后迅速宣布启动“星际之门”计划,预计在未来几年内投资高达5000亿美元用于建设人工智能基础设施,特别是超大规模数据中心集群\footnote{郑执浩:《美国人工智能建设的新发展:“星际之门”计划前景分析》,《现代国际关系》2025年第4期,第25-26页。}。该计划并非单纯的商业投资行为,而是具有极其明显的国家战略动员色彩。它采取了典型的公私合作模式,由私营科技巨头提供资金并主导技术运营,而联邦政府则通过土地使用许可、税收减免、放松环境监管以及简化电网接入审批流程等方式,提供全方位的政策便利。这种紧密的“政府主导-科技资本”联系,体现了美国试图通过超常规的大规模基础设施建设,确立其在人工智能底层算力领域的绝对霸权。

为配合算力层面的物理整合,美国军方在数据基座的重组上也采取了激进的举措。2025年11月,特朗普签署第14363号行政令,正式启动“创世纪任务”。该计划明确要求掌握由人工智能驱动的科学发现能力,由能源部牵头建立统一的“美国科学与安全平台”,强制整合联邦政府数十年来积累的庞大科学数据集,并统筹调配包括橡树岭、阿贡在内的国家实验室超级计算资源\footnote{Executive Order 14363, "Launching the Genesis Mission," Federal Register, Vol. 90, No. 227, November 28, 2025, pp. 55035-55037.}。

紧接着在2026年初,美国战争部常务副部长签署备忘录,宣布将原有的数据平台进行根本性重组,集中打造“战争数据平台”。这一举措旨在打破军种间长期存在的数据壁垒,构建一个能支撑智能体在全军范围内快速开发与集成的底层数据枢纽,确保从财务审计、后勤调度到前端杀伤链的每一个环节,都运行在统一、受控且高度标准化的数据基座之上\footnote{Deputy Secretary of War, "Memorandum: Transforming Advana to Accelerate Artificial Intelligence and Enhance Auditability," Washington, D.C., January 9, 2026, pp. 1-2.}。这种将数据要素高度国有化、算力资源绝对集中化的做法,实质上是美国政府试图垄断科学发现与战争决策的底层逻辑,从而建立起对他国的绝对代差优势。

\subsection{技术标准的“去伦理化”与敏捷响应机制}

在传统的大国军事竞争中,庞大而僵化的官僚体制和繁琐的采办流程往往是阻碍新技术向实际战斗力转化的最大绊脚石。为支撑高强度的科技竞赛,特朗普政府大力推动体制改革,试图赋予庞大的政府和军事机器以硅谷初创企业般的敏捷性。

在宏观政策导向上,特朗普政府推行“创新优先、监管最小化”的逻辑。2025年1月签署的第14179号行政令,明确废除了拜登政府时期关于安全、可靠地开发和使用人工智能的行政令,旨在消除被其视为阻碍创新的监管障碍\footnote{马贺非、毛维准:《特朗普政府人工智能基础设施政策探析》,《现代国际关系》2025年第4期,第8-9页。}。这种“去伦理化”的政策倾向,反映了其将竞争优势置于安全考量之上的实用主义考量,试图通过放松监管来加速技术的野蛮生长。

在军事采办层面,美国战争部在2025年大力推动规划、计划、预算与执行体制改革\footnote{Under Secretary of Defense (Comptroller), "Planning, Programming, Budgeting, and Execution Reform Implementation Report," January 16, 2025, pp. 1-4.},并于2026年初完成了国防创新生态的根本性重组。2026年1月发布的《国防创新生态系统转型》备忘录,确立了负责研究与工程的副部长作为战争部唯一的“首席技术官”,统摄全局技术方向。同时,将国防创新单元和战略能力办公室升格为外勤活动机构,赋予其独立的人事与预算权限,旨在彻底消除多头管理的内耗\footnote{Secretary of War, "Memorandum: Transforming the Defense Innovation Ecosystem to Accelerate Warfighting Advantage," Washington, D.C., January 9, 2026, pp. 1-2.}。

这一体制改革配合了极具攻击性的预算机制创新。依托特朗普签署的《大美法案》所提供的巨额资金支持,战争部新设立了“创新插入增量”预算机制,并成立了由首席技术官领导的“障碍移除委员会”。该委员会被赋予了极高的特权,能够快速豁免非强制性的法规限制,绕过繁琐的审批程序\footnote{Ibid., p. 4.}。这一系列制度设计的明确目标,在《战争部人工智能战略》中被量化为:在商业前沿模型公开发布的三十天内,完成其在美军作战系统中的部署\footnote{Secretary of War, "Memorandum: Artificial Intelligence Strategy for the Department of War," Washington, D.C., January 9, 2026, p. 4.}。这折射出特朗普政府试图通过制度创新,将商业技术优势无缝转化为即时安全优势的战略企图。

\subsection{“硅基和平”与全球产供应链的阵营化重塑}

在确保自身技术研发与基础设施建设绝对领先的同时,特朗普政府在国际舞台上推出了具有强烈扩张性和排他性的技术输出与供应链重塑战略,意图在全球范围内划定数字势力范围。

一方面,美国试图通过出口管制等强制性技术权力,限制竞争对手获取先进技术资源。2025年,美国商务部工业与安全局发布新规,对高端人工智能芯片的管制进一步升级,并引入“三级穿透信息”披露的审查方式,追溯人工智能技术的军民两用风险\footnote{郭畅、冯晓青:《进攻性技术民族主义与特朗普2.0时期的美国对华人工智能战略》,《太平洋学报》2025年第11期,第65-66页。}。此外,美国还通过“芯片四方联盟”等机制,施压盟友在技术投资、标准制定上配合其出口管制政策,试图从底层硬件基础上遏制中国人工智能的发展。

另一方面,美国推行《促进美国人工智能全栈技术出口计划》,向全球盟友及部分发展中国家推销包含优化硬件、数据通道、云服务及底层大模型在内的整体解决方案\footnote{Executive Order 14320, "Promoting the Export of the American AI Technology Stack," Federal Register, Vol. 90, No. 142, July 28, 2025, pp. 35393-35394.}。美国总统经济顾问委员会在报告中首次披露了名为“硅基和平”的国际供应链联盟。该联盟试图将处于供应链上游的设备制造国与下游拥有雄厚资本和能源的数据中心投资国强行捆绑在美国的技术架构内\footnote{The Council of Economic Advisers, "Artificial Intelligence and the Great Divergence," Executive Office of the President, January 2026, p. 6.}。

这种策略的险恶之处在于其深刻的路径依赖与主权剥夺效应。一旦某国的数字底座由美国标准的硬件和闭源协议构成,其后续的软件生态繁衍、数据流转监管乃至国家安全体系的运行,将被迫对美国单向透明。这实际上是美国在二十一世纪推行的数字新殖民主义,企图通过物理基础设施的排他性,将以中国为代表的竞争对手从全球关键地缘节点的数字生态中物理剥离,强行降下一道割裂世界的数字铁幕。

\section{认知域重塑:智能体驱动的认知对抗与话语霸权}

进入特朗普2.0时期,美国政府的科技战略已完全超越了硬实力竞争的传统范畴,转而将科技优势特别是大语言模型的话语生成能力,异化为重塑全球认知结构、界定国际规则的底层工具。认知战已被美国从辅助性的战术手段,正式提升为核心的主战样式,力图构建服务于美国优先的全球认知闭环。

\subsection{智能体驱动的认知对抗机制升级}

美国军事战略文化正在经历从隐蔽的信息干预向公开的、进攻性心理操纵的重大转型。《2025国家安全战略》明确指出,过去数十年的国家建设和乌托邦式理想主义侵蚀了美军的战士精神,必须予以全面恢复。

在此宏观指导下,2025年12月,美国战争部发布了一份极具冷战色彩的备忘录,正式废弃了沿用多年的、显得相对温和的“军事信息支援行动”称谓,高调恢复使用“心理战”这一术语。该文件毫不掩饰地指出,美国军方将致力于实施旨在影响外国政府、组织、群体及个人的情感、动机、客观推理,并最终改变其行为的计划行动\footnote{Secretary of War, "Memorandum: Changing the Term Military Information Support Operations Back to Psychological Operations," Washington, D.C., December 2, 2025.}。

这一战略回归绝非仅仅是字面意义上的更名,而是在《战争部人工智能战略》中得到了极其恐怖的技术支撑。该战略启动了七大领跑项目,其中“蜂群工场”和“智能体网络”项目,旨在利用智能体直接参与从战役规划到杀伤链执行的决策闭环;而“开放兵工厂”项目则致力于将海量情报在数小时而非数年内转化为武器化的代码与认知攻击素材\footnote{Secretary of War, "Memorandum: Artificial Intelligence Strategy for the Department of War," Washington, D.C., January 9, 2026, pp. 2-3.}。这意味着美军已将操纵对手国民认知、瓦解敌方抵抗意志视为与物理火力打击同等重要的独立作战能力。

随着智能体技术的应用,认知对抗的模式发生了显著升级。智能体具备自主感知环境、制定策略并采取行动的能力,使得认知干预从传统的单向信息传播,演变为“感知-决策-行动”闭环的动态交互过程\footnote{韩娜、邹初妤:《智能认知对抗:理论演化、对抗机制与安全风险》,《国际安全研究》2025年第6期,第52-53页。}。在智能体的加持下,认知对抗的规模、速度和隐蔽性都将大幅提升,认知域的灰度竞争将迅速向高烈度对抗演变。

\subsection{算法价值观投射与制度性权力扩张}

在数字时代,大语言模型的伦理标准和对齐机制直接决定了人类社会的信息筛选与历史叙事。特朗普政府敏锐地意识到,人工智能模型并非价值中立的工具,而是蕴含着意识形态能量的载体。美国内部各州不一的监管法律,甚至硅谷科技巨头内部偏向左翼的多元文化主义,正在严重削弱美国对外输出统一认知叙事的能力。

为了消除这种内部噪音,确立符合其战略利益的认知标准,特朗普政府采取了雷厉风行的行政干预手段。2025年12月签署的第14365号行政令《确保国家政策框架》中,特朗普政府动用了联邦优先权,授权美国司法部长成立诉讼特别工作组,专门负责起诉和打压如科罗拉多州等试图立法禁止算法歧视的地方州政府\footnote{Executive Order 14365, "Ensuring a National Policy Framework for Artificial Intelligence," Federal Register, Vol. 90, No. 239, December 16, 2025, pp. 58499-58500.}。

同时,该行政令还动用联邦宽带资金作为要挟,迫使各州放弃对模型输出内容进行多元化约束。结合战争部对采购中“任何合法用途”的强制要求,美国正在通过国家机器强行清洗模型中的觉醒文化参数,建立一套符合右翼保守主义价值观的统一算法标准。当这套经过所谓净化与去意识形态化包装的算法标准,随着全栈出口计划推向全球时,实际上是在向接收国输出一种极具隐蔽性、排他性和同化能力的认知霸权。数字巨头作为技术权力的实际行使者,通过控制知识的生产、传播以及分配规则,影响其他行为体的认知框架与决策空间\footnote{孙志伟、殷浩铖:《人工智能时代数字巨头的技术权力及其对“全球南方”的挑战》,《国际安全研究》2025年第2期,第148-149页。}。这将在潜移默化中重塑全球南方国家民众的历史观与价值观。

\subsection{信息污染与全球认知主权的系统性侵蚀}

生成式人工智能的高效信息生成能力,为认知战提供了低成本制造海量信息的技术手段。这不仅提升了认知战的效能,也带来了信息污染和认知主权侵蚀的严重风险。

在认知战中,进攻方可以使用生成式人工智能工具制造大量虚假信息或具有特定倾向性的误导信息。这些信息在社交媒体等网络空间中快速传播,容易形成“信息茧房”和“回声室效应”,阻碍受众获取真实、全面的信息\footnote{贾子方、王栋:《生成式人工智能对国际安全的影响:以认知战为路径的分析》,《国际政治研究》2025年第3期,第91-92页。}。长期的信息污染不仅会扭曲个体的认知判断,还可能加剧社会群体的认知分裂和对立情绪,进而引发社会信任危机和政治动荡。

此外,认知战的终极防线不仅在于当下的舆论场,更在于未来的人才与思想阵地。特朗普2.0时期将国家安全观念的灌输与技术标准的同化,前推至基础教育阶段,试图从代际源头巩固美国的认知优势。第14277号行政令《推进美国青少年的教育》提出建立白宫教育特别工作组,并举办覆盖全国的总统挑战赛\footnote{Executive Order 14277, "Advancing Artificial Intelligence Education for American Youth," Federal Register, Vol. 90, No. 80, April 28, 2025, pp. 17519-17520.}。这一政策的深层逻辑在于通过联邦政府的直接介入与资金引导,将国家安全需求、美式技术伦理与基础教育深度融合。这种从技术底层到教育源头的全方位认知塑造,对其他国家的认知主权和文化安全构成了系统性挑战。
\section{特朗普2.0时期人工智能战略的内在悖论与实施困境}

特朗普2.0政府通过“星际之门”计划重塑算力基建,以及将认知对抗系统化等一系列举措,试图构建起一张精密的技术与认知封锁网。这不仅对战后以联合国为核心的多边秩序造成了颠覆性冲击,更在物理与精神双重层面对我国构成了空前严峻的系统性战略挤压。然而,这一基于进攻性技术民族主义的战略,其排他性、封闭性与零和博弈内核,与全球化时代科技创新的开放性、协同性相冲突。在实践中,该战略不可避免地陷入系统性困境,面临着多重内在悖论与外部挑战。

\subsection{封闭垄断与开源创新的结构性冲突}

进攻性技术民族主义驱使美国采用技术封锁与规则排他等手段压制他国创新。然而,这一逻辑与现代技术创新规律存在深刻矛盾。人工智能的演进高度依赖全球性创新网络、海量数据与多样化场景,封闭体系最终可能拖慢其自身技术迭代速度。

中国在人工智能领域的非对称创新突破,凸显了美国封锁战略的局限性。面对美国的算力与芯片管制,中国企业并未完全陷入被动,而是通过算法优化、场景创新与开源生态建设,探索出一条不同的技术路径。例如,中国企业推出的开源模型,以极低的算力投入,在多项关键基准测试中展现出比肩甚至超越西方前沿模型的性能\footnote{鲁传颖、才悦:《特朗普2.0时期中美人工智能博弈的新阶段》,《太平洋学报》2025年第10期,第21页。}。这种“算力约束下的模型优化”路径,证明了开源模式在汇聚全球智慧、降低创新门槛方面的巨大潜力。

中国开源模型的崛起,不仅打破了美国科技巨头对人工智能模型的垄断,也对美国试图构建的封闭技术生态形成了有力冲击。这表明,单纯的技术封锁难以遏制具有完备工业体系和庞大人才基数的大国崛起,反而可能激发其自主创新的内生动力,加速全球技术体系的多极化发展。

\subsection{算力扩张的能源瓶颈与复合体的自限性}

美国试图通过“星际之门”等超大规模基础设施项目确立绝对的算力优势,但这一宏大计划在实施过程中面临着严峻的现实制约。

首先是能源供给的瓶颈。生成式人工智能数据中心是名副其实的“耗电巨兽”。据预测,到2030年,美国数据中心的电力需求将大幅增长,占全国总电力需求的比例将显著上升\footnote{马贺非、毛维准:《特朗普政府人工智能基础设施政策探析》,《现代国际关系》2025年第4期,第19-20页。}。然而,美国老化的电网设施、漫长的能源项目审批流程以及对传统化石能源的依赖,使得电力供应的稳定性和可持续性面临巨大挑战。能源缺口可能成为制约美国人工智能基础设施扩张的软肋。

其次是“科技-工业复合体”内部的利益分歧与自限性。美国人工智能基础设施的建设高度依赖少数科技巨头。虽然这些企业在推动技术创新方面发挥了重要作用,但其逐利本性也导致了复合体内部的激烈竞争与利益冲突。例如,在“星际之门”项目中,参与方在技术路线、资本合作及数据中心运营等方面存在潜在的分歧\footnote{郑执浩:《美国人工智能建设的新发展:“星际之门”计划前景分析》,《现代国际关系》2025年第4期,第28-29页。}。此外,过度依赖少数企业可能导致技术发展路径的僵化,削弱国家整体的创新韧性。

\subsection{盟友战略自主诉求与“数字铁幕”的推行阻力}

美国试图通过联盟体系和“硅基和平”等排他性机制,在全球范围内推行其技术标准和出口管制政策。然而,这一战略在实施中遭遇了来自盟友及全球南方国家的普遍抵制。

一方面,美国将联盟工具化、交易化的做法,忽视了盟友的经济利益与战略自主性。美国施压盟友对华实施先进技术禁售,直接冲击了相关国家企业的市场份额,引发了盟友内部的不满。在经济利益和战略自主的考量下,欧盟、日本等关键行为体在对华技术政策上与美国产生分歧,纷纷推进产业自主计划,以降低对美技术依赖\footnote{郭畅、冯晓青:《进攻性技术民族主义与特朗普2.0时期的美国对华人工智能战略》,《太平洋学报》2025年第11期,第70页。}。

另一方面,广大全球南方国家对美国试图构建的“数字铁幕”保持高度警惕。发展中国家渴望搭乘人工智能技术发展的快车,但不希望沦为大国竞争的棋子或数字附庸。美国附加政治条件的技术输出,引发了这些国家对数字主权丧失的深层担忧。因此,美国难以构建起铁板一块的遏华技术联盟,其试图垄断全球人工智能价值链的战略目标面临重重阻力。

\section{跨越“数字铁幕”:中国的战略应对与多边主义实践}

面对特朗普2.0政府在人工智能领域构筑的技术与认知防线,中国需准确研判其战略意图。在保持战略定力的同时,应统筹发展与安全,采取灵活务实的应对之策,以真正的多边主义和高水平科技自立自强,化解外部战略挤压。

\subsection{坚持高水平科技自立自强,构筑开源创新生态}

针对美国的技术封锁与“敏捷突袭”风险,中国必须将科技自立自强作为国家发展的战略支撑。应充分发挥新型举国体制优势,加强原始创新和关键核心技术攻关,在人工智能芯片、关键算法等底层技术上实现非对称突破。

同时,应大力倡导和构筑开源创新生态。开源模式不仅能够汇聚全球开发者的智慧,加速技术迭代,还能有效规避单一边界的技术封锁。中国应积极参与并主导国际开源社区的建设,支持国内优秀的人工智能模型开源开放,为全球特别是广大发展中国家提供低成本、高性能的技术选项。通过构建“底层技术-应用场景-资本支持”的完整生态,形成具有国际竞争力的自主技术体系,从根本上打破美国的技术垄断。

\subsection{携手“全球南方”践行真正的多边主义}

美国推行的排他性技术输出,附带极强的政治条件,这恰恰触及了全球南方国家的战略痛点。中国应抓住这一契机,全面升级技术出海策略,从单纯的产品输出转向全方位的“技术赋能”。

依托《全球人工智能治理倡议》和全球发展倡议,中国应高举“数字主权”和“技术不结盟”的旗帜,向全球南方国家提供去捆绑、开源可控的替代性人工智能与算力基础设施方案\footnote{李猛:《中国携手“全球南方”倡导践行真正的多边主义的实践与展望》,《南亚研究》2025年第3期,第21页。}。通过帮助发展中国家建立本土化的数据中心、培养本土技术人才,中国可以助力其跨越“数字鸿沟”。在联合国等多边框架下,中国应与全球南方国家一道,共同参与人工智能国际规则的制定,推动全球人工智能治理朝着更加公平、包容、普惠的方向发展,构建广泛的国际数字统一战线。

\subsection{统筹发展与安全,构建国家认知安全与社会韧性体系}

面对智能化认知对抗带来的政治安全、社会极化与文化侵蚀风险,中国必须从总体国家安全观的高度,构建全面、动态的认知安全治理体系。

在技术层面,应超前布局认知安全关键技术,建立从感知、阻断到主动防御的智能化防御链。强化对深度伪造、自动化虚假信息生成等认知攻击手段的识别与反制能力。在制度层面,应完善人工智能相关法律法规,明确平台与开发者的责任边界,设立认知风险影响评估制度。在社会层面,应系统性构建社会认知韧性。通过国民教育强化全民特别是青少年的信息素养与批判性思维,提升公众在复杂信息环境下的辨别力与“认知免疫力”\footnote{韩娜、邹初妤:《智能认知对抗:理论演化、对抗机制与安全风险》,《国际安全研究》2025年第6期,第67页。}。同时,加快培养兼具前沿技术背景与国际政治视野的复合型战略人才,为应对未来的认知域对抗提供坚实的人才保障。

\section*{结语}
\addcontentsline{toc}{section}{结语}

特朗普2.0政府在人工智能领域的战略转型,是一次试图通过行政手段和国家资本力量,强行逆转多极化历史周期的大国博弈。其核心逻辑在于透支现存的技术存量与金融霸权,通过扭曲市场规则、重塑科学范式与输出特定价值观,构建一个排他性的“科技-认知”闭环体系,以期维持其全球霸权地位。

然而,从长远的历史视角审视,这种基于进攻性技术民族主义的战略不可避免地带有内在的脆弱性与逻辑悖论。试图通过行政权力垄断技术发展路径,背离了科技创新依赖开放与共享的客观规律;将算法标准异化为霸权工具,难以获得国际社会的广泛认同,反而会激发全球范围内的防御性反弹。

对于中国而言,面对外部的“数字铁幕”与认知攻势,应始终保持高度的战略定力与制度自信。历史经验表明,单纯的技术封锁难以遏制一个拥有完备工业体系和庞大创新潜力的大国的崛起。只要中国坚持高水平科技自立自强,坚定维护国家数字主权,积极践行真正的多边主义,持续拓展平等互利、包容普惠的国际技术合作网络,就一定能在百年未有之大变局中掌握战略主动,为推动构建人类命运共同体贡献中国智慧与中国力量。

\end{document}
