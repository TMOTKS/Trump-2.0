\documentclass[12pt,a4paper,fontset=fandol]{ctexart}

% 引入必要的宏包
\usepackage[margin=2.5cm]{geometry} % 设置页边距
\usepackage{setspace} % 设置行距
\usepackage{titlesec} % 标题格式设置
\usepackage[colorlinks=true,linkcolor=black,urlcolor=blue,citecolor=black]{hyperref} % 超链接
\usepackage{indentfirst} % 首行缩进

% ==========================================
% 核心修复:将脚注标号修改为 [1], [2] 格式
% ==========================================
\renewcommand{\thefootnote}{[\arabic{footnote}]}

% 格式设置:符合国内学术期刊规范
\onehalfspacing % 1.5倍行距
\titleformat{\section}{\Large\bfseries\centering}{\thesection}{1em}{} % 一级标题居中
\titleformat{\subsection}{\large\bfseries}{\thesubsection}{1em}{}
\renewcommand{\thesection}{\chinese{section}、}
\renewcommand{\thesubsection}{(\chinese{subsection})}

\pagestyle{plain} % 去除页眉

\title{\textbf{科技霸权与认知重塑——特朗普2.0时期对华人工智能战略的逻辑与应对}}
\author{}
\date{}

\begin{document}

\maketitle

\begin{abstract}
\noindent \textbf{【内容摘要】} 人工智能已成为大国战略竞争的核心场域。2026年初,随着特朗普第二任期各项政策的密集落地,美国对华科技战略已从选择性脱钩升级为以“进攻性技术民族主义”为内核的系统性遏制。在《2025国家安全战略》和《2026国家国防战略》的宏观指导下,特朗普政府通过启动“创世纪任务”、实施“战争部全面转型为AI优先作战力量”及重塑国防创新生态,构建了一套集物理控制与认知塑造于一体的战略体系。该体系在硬实力层面,试图通过国家意志强力调配算力与数据资源,建立垄断性的科研与采办体系;在软实力层面,将心理战(PSYOP)制度化,并试图通过“反觉醒”技术标准的输出确立排他性的认知霸权。本文基于2025年至2026年初美国联邦政府发布的关键行政令和政策档案,系统梳理特朗普2.0政府科技与认知新政的实施路径,剖析其背后的权力运作逻辑,并提出我国的战略应对建议。

\vspace{1em}
\noindent \textbf{【关键词】} 特朗普2.0;人工智能;进攻性技术民族主义;认知战;大国竞争;数字主权
\end{abstract}

\vspace{2em}

\section*{引言}
\addcontentsline{toc}{section}{引言}

当前,世界正经历百年未有之大变局,大国战略竞争与第四次科技革命相互交织。人工智能技术因其高度的战略性、基础性与军民两用性,已超越单纯的技术属性,成为重塑全球经济、军事与权力结构的关键变量。在此背景下,科技创新深度嵌入地缘政治博弈,人工智能安全逐渐成为国家安全体系的核心议题。

2025年至2026年初,随着特朗普再次当选美国总统,美国大战略发生深刻质变。如果说特朗普首个任期的显著特征是粗放式的关税壁垒与实体贸易脱钩,那么其第二个任期则在汲取前期经验的基础上,展现出更为精细且具系统性的重塑特征。华盛顿决策层不再满足于维持经济指标的表层领先,而是致力于构建一个由美国主导、排斥战略竞争对手的“技术-认知”闭环体系。
\section{认知域重塑:心理战的制度化与霸权输出}

进入特朗普2.0时期,美国政府的科技战略已超越硬实力竞争范畴,转而将科技优势异化为重塑全球认知结构、界定国际规则的底层工具。认知战已被美国从辅助性的战术手段,正式提升为核心的主战样式,力图构建服务于“美国优先”的全球认知闭环。

\subsection{心理战的战略回归与“AI智能体”战术升级}

美国军事战略文化正在经历从防御性威慑向进攻性实战的重大转型。《2025国家安全战略》明确指出,过去数十年的“国家建设”和“乌托邦式理想主义”侵蚀了美军的“战士精神”,必须予以恢复\footnote{President of the United States, \textit{National Security Strategy of the United States of America}, The White House, November 2025.}。

在此指导下,2025年12月2日,战争部发布备忘录,废弃沿用多年的“军事信息支援行动”称谓,恢复使用“心理战”这一冷战色彩浓厚的术语,明确指出旨在“影响外国政府、组织及个人的情感、动机和客观推理”\footnote{Secretary of War, \textit{Memorandum: Changing the Term Military Information Support Operations Back to Psychological Operations}, Washington, D.C., December 2, 2025.}。

这一战略回归在2026年1月发布的《战争部AI战略》中得到了具体的科技支撑。该战略明确提出将美军打造为“AI优先”部队,并启动了七大“领跑项目”。其中,“蜂群工场”和“智能体网络”项目,旨在利用AI智能体直接参与从战役规划到杀伤链执行的决策闭环;而“开放兵工厂”项目则试图将情报在数小时内转化为武器化代码\footnote{Secretary of War, \textit{Memorandum: Artificial Intelligence Strategy for the Department of War}, Washington, D.C., January 9, 2026.}。这意味着美军已将操纵对手认知视为与火力打击同等重要的独立作战能力。

\subsection{联邦优先权下的“反觉醒”算法标准统一}

在数字时代,算法的伦理标准直接决定了信息的筛选与呈现机制。特朗普政府意识到,美国内部各州不一的AI监管法律甚至左翼的多元文化主义,正在削弱其对外输出统一认知叙事的能力。

在此基础上,战争部在《战争部AI战略》备忘录中确立了“负责任AI”的新标准,明确禁止使用经过“意识形态调整”的AI模型,要求模型必须提供“客观真实”的回应,并明令禁止“DEI”等意识形态干扰AI的军事应用\footnote{Secretary of War, \textit{Memorandum: Artificial Intelligence Strategy for the Department of War}, Washington, D.C., January 9, 2026.}。

同时,结合此前第14365号行政令建立的“AI诉讼特别工作组”\footnote{Executive Order 14365, \textit{Ensuring a National Policy Framework for Artificial Intelligence}, Federal Register, Vol. 90, No. 239, December 16, 2025.},以及第14179号行政令关于消除AI开发中“觉醒文化”障碍的要求\footnote{Executive Order 14179, \textit{Removing Barriers to American Leadership in Artificial Intelligence}, Federal Register, Vol. 90, No. 20, January 31, 2025.},美国正在通过行政干预清洗AI模型中的多元文化主义参数,建立一套符合右翼保守主义价值观的统一算法标准。这种通过行政令强行统一技术标准的做法,旨在确保美国对外输出的AI模型具备高度的认知一致性,向接收国输出一种隐蔽的认知霸权。

\subsection{教育体系的军事化重塑与代际认知预置}

认知战的终极防线在于人才与未来的思想阵地。特朗普2.0时期将国家安全观念的灌输前推至基础教育阶段,试图从代际源头巩固美国的认知优势。

第14277号行政令《推进美国青少年的AI教育》明确提出建立“白宫AI教育特别工作组”并举办“总统AI挑战赛”\footnote{Executive Order 14277, \textit{Advancing Artificial Intelligence Education for American Youth}, Federal Register, Vol. 90, No. 80, April 28, 2025.}。该政策超越了提升青少年数字技能的一般性教育目标,其深层逻辑在于通过联邦政府的直接介入,将国家安全需求、美式技术伦理与基础教育深度融合。通过在中小学阶段普及符合国家战略需求的AI课程,美国试图培养一支对美式技术标准有天然亲和力、对国家战略有深刻认同感的后备军。这种战略布局,是在为未来长期的大国认知博弈进行深度的兵力预置。

\section{“新曼哈顿”战略对国际秩序与我国的系统性影响}

特朗普2.0政府通过“创世纪任务”垄断科研范式、利用“创新生态重组”提升资金敏捷性,以及将心理战制度化等一系列举措,已构建起一张精密的技术与认知封锁网。这不仅对我国的科技发展构成物理阻断,更在国际舆论场和规则制定权上形成系统性压制。

\subsection{加剧“同时性问题”与“敏捷性突袭”风险}

美国战争部新设立的“首席技术官”体制及“领跑项目”机制,将极大提升美军的资源调配与技术转化速度。备忘录中明确提出的“30天内部署最新模型”及“障碍移除委员会”的设立,意味着美军将具备像软件迭代一样快速更新装备的作战能力。

这将彻底改变传统的战争准备节奏,加剧《2026国家国防战略》中提到的“同时性问题”\footnote{Secretary of War, \textit{2026 National Defense Strategy}, Department of War, January 23, 2026.},即美军试图通过AI赋能,同时在多个战区保持高强度介入能力。这种对“速度制胜”的追求,增加了大国间发生“敏捷性突袭”和战略误判的风险。

\subsection{技术生态阵营化与“数字铁幕”的实质性降临}

特朗普政府推行的“全栈技术出口”计划,本质上是一种数字殖民主义。通过物理层锁定和标准捆绑,美国试图剥夺发展中国家的数字主权。

《2025国家安全战略》明确指出,美国不仅要控制西半球的政治安全,更要通过“招募与扩张”策略,控制关键供应链、矿产资源和数字基础设施,防止外部竞争者获得战略立足点\footnote{President of the United States, \textit{National Security Strategy of the United States of America}, The White House, November 2025.}。这种策略实际上是在全球范围内划定“数字势力范围”,通过物理基础设施的排他性,企图将竞争对手从全球关键地缘节点的数字生态中物理剥离,强行降下一道“数字铁幕”。

\subsection{国际舆论场与规则制定权的系统性压制}

在认知域,美国将算法标准异化为霸权工具。通过行政手段清除AI模型中的多元包容参数,确立了带有强烈右翼保守主义色彩的底层算法逻辑。当这套经过“净化”的算法标准推向全球时,将在潜移默化中重塑全球受众的信息获取习惯与客观推理逻辑。这不仅是对我国国际话语权的隐蔽剥夺,更是对全球文明多样性的系统性压制。

\section{跨越“数字铁幕”:我国的战略应对与路径选择}

面对特朗普2.0政府构筑的“科技-认知”闭环体系,我国需准确研判其战略意图,在保持战略定力的同时,采取灵活务实的应对之策。

\subsection{加快国防采办与科研体制的适应性改革}

针对美军“快速采办”和“非程序化资金流动”带来的敏捷突袭风险,我国不能再仅以传统的年度预算周期来预判对手的行动速度。建议建立动态监测预警机制,重点关注其利用灵活资金在认知域和其它作战领域可能发动的突袭。同时,应加快自身的国防采办与科研管理体制改革,探索建立适应智能化战争需求的快速响应机制,确保在技术对抗中不因制度摩擦而丧失时间窗口。

\subsection{构建“数字主权”统一战线,反制技术殖民}

特朗普政府的排他性技术输出附带着极强的主权干涉,这为我国提供了反制空间。广大“全球南方”国家虽渴求技术发展,但并不希望沦为任何大国的数字附庸,对数据主权和技术独立有着天然诉求。

我国应抓住这一战略痛点,调整技术出海策略,从单纯的产品输出转向“技术赋能”。倡导“数字主权”和“技术不结盟”理念,向全球南方国家提供去捆绑、可控、独立的替代性技术方案,帮助其建立自主的数字基础设施。通过支持各国维护自身的数据主权和算法独立,构建广泛的国际数字统一战线,以此对冲美国的排他性技术霸权。

\subsection{利用美国联邦体制裂痕实施精准反制}

特朗普2.0时期激进的联邦集权政策已在美国国内引发剧烈反弹。第14365号行政令对州级立法权的压制,以及在《战争部AI战略》中对“觉醒文化”的公开清洗,使得联邦政府与自由派重镇及硅谷科技界之间的裂痕空前扩大。

在应对美国的认知战攻势时,我国应善于发现并利用其内部矛盾。在对外传播中,客观揭露其“联邦集权”对美国自身民主多元价值观的破坏,以及技术霸权对全球数字生态多样性的威胁。同时,可加强与美国地方州、科技界理性力量及民间社会的交流合作,利用美国联邦体制的缝隙,通过务实的经贸与文化纽带对冲华盛顿的极端政策。

\subsection{强化认知防御与复合型人才培养}

面对美国将认知战前推至基础教育领域的长远布局,我国必须从国家安全高度审视教育与人才培养体系。单纯的理工科教育已不足以应对未来的混合战争,必须强化全民特别是青少年的认知防御能力。

建议在国民教育体系中加强算法伦理、信息溯源及认知安全教育,提升公众在复杂信息环境下的辨别力与免疫力。同时,应加快培养兼具技术背景与国际政治视野的复合型人才,建立一支既懂前沿技术又精通心理博弈的专业队伍,以应对日益复杂的认知域对抗,防止在代际竞争中出现人才与认知断层。

\section*{结语}
\addcontentsline{toc}{section}{结语}

特朗普2.0政府的新政,是一次试图通过行政手段强行逆转历史周期的战略尝试。其核心逻辑在于利用美国现存的技术存量与金融霸权,通过扭曲市场规则与重塑科学范式,构建一个排他性的“科技-认知”闭环体系。这种做法虽在短期内可能通过物理隔绝与认知操纵为美国带来战术优势,但从长远历史视角审视,其内在的脆弱性已然显现。

首先,这种“新曼哈顿”式的科技动员体系试图通过行政权力强行垄断科学发现的解释权,本质上背离了科技创新所依赖的开放与多元精神,长期来看恐将导致科研生态的僵化与创新活力的枯竭。其次,认知战的制度化与进攻性转向,在提升美军作战效能的同时,也加速了美国软实力的道德透支。当“反觉醒”和“去意识形态化”成为新的“政治正确”时,这种将算法标准异化为霸权工具的做法,难以获得国际社会的广泛认同,反而会刺激“全球南方”国家对于数字主权和信息安全的深层担忧,引发全球范围内的防御性反弹与技术“去美化”进程。最后,美国国内政治的结构性矛盾将成为制约其战略目标实现的内生阻力。联邦集权与地方分权、右翼保守主义与多元自由主义之间的深刻裂痕,无法通过行政命令强行弥合。

对于我国而言,面对特朗普2.0政府构筑的“数字铁幕”与认知攻势,应保持战略定力,既不盲目焦虑,亦不掉以轻心。历史经验表明,单纯的技术封锁难以遏制大国崛起,认知的操纵也终将被客观事实解构。只要坚持科技自立自强,坚定维护国家数字主权,积极拓展平等互利的国际技术合作网络,便能有效对冲外部风险,在变局中掌握战略主动。

\end{document}


在这一新兴的战略体系中,前沿科技特别是生成式人工智能,被异化为政治权力和地缘博弈的延伸;与此同时,认知领域则被美国军方及情报界定义为继陆、海、空、天、网之后的“第六作战域”,是大国博弈着力争夺的战略高地。面对这一历史性的新变局,深入剖析特朗普政府在科技动员、认知重塑及规则锁定方面的政策文本,厘清其从物理层面的技术隔离向精神层面的认知霸权升级的战略逻辑,对于准确研判未来国际秩序走向、防范系统性战略风险,以及构建我国的战略应对体系,具有重要的现实意义与理论价值。

\section{特朗普2.0时期人工智能战略的深层动因与逻辑}

特朗普2.0政府在2025年底至2026年初密集出台极具进攻性的科技与认知政策,并非出于偶然的政策冲动,而是基于其核心幕僚团队对国际权力格局演变、国内政治极化态势以及智能化战争形态的综合研判。这种战略转型具有深刻的内生动因与底层逻辑。

\subsection{进攻性技术民族主义与霸权护持}

随着多极化趋势的深入发展,美国战略界对自身综合国力相对衰落的焦虑感明显上升。特朗普2.0政府的经济与安全智囊认为,传统的自由市场机制无法在当前的大国竞争中确保美国在关键技术领域的绝对领先地位。因此,美国政策取向明显向“进攻性技术民族主义”倾斜,试图以非对称的技术代差来重塑全球权力分配。

为实现这一目标,美国打破了传统的小政府与自由放任理念,转而利用国家意志强行干预技术生态。在《2026国家国防战略》(National Defense Strategy)中,美国战争部明确将“重振国防工业基础”列为四大核心路线之一,并直白地提出必须进行“如同世界大战期间的国家工业动员”\footnote{Secretary of War, \textit{2026 National Defense Strategy}, Department of War, January 23, 2026.}。这种基于霸权焦虑的国家级动员逻辑,促使美国试图通过行政命令强行打通联邦数据、国家实验室算力与私营部门模型之间的壁垒,构建起一套举国体制下的科技动员体系,意在从源头垄断未来技术的发展路径。

\subsection{“科技-工业复合体”的利益驱动与“内政外化”}

特朗普2.0战略的另一显著特征,是国内政治生态对外交与科技政策的深刻形塑。近年来,美国国内政治愈发成为影响其外交政策的关键变量,“内政驱动外交”的态势日益凸显。

在这一逻辑下,硅谷科技精英与右翼政治力量的合流发挥了关键作用。特朗普政府将国内的右翼保守主义与“身份政治”诉求,强行延伸至国际战略和技术标准领域。例如,《2025国家安全战略》(National Security Strategy)明确将“能力与优绩主义”列为核心原则,公开反对“DEI”(多元、公平、包容)等被其视为“反竞争”的激进意识形态\footnote{President of the United States, \textit{National Security Strategy of the United States of America}, The White House, November 2025.}。这种“科技-工业复合体”的深度内嵌,不仅推动了美国人工智能政策向重创新、轻监管的方向倾斜,也使得技术议题不可避免地被意识形态化和安全化,成为转移国内矛盾、凝聚选民基础的政治工具。

\subsection{智能化战争形态演进与本体安全焦虑}

生成式人工智能与智能体(Agentic AI)技术的突破性进展,为特朗普政府重塑军事与认知战略提供了现实驱动力。美国战争部深刻认识到,技术赋能的战争模式和武器能力开发,将在未来十年重新定义军事事务的特征与战争的制胜机理。

在此背景下,特朗普政府确立了将美军全面转型为“AI优先”(AI-first)作战力量的宏大目标。根据2026年1月发布的《战争部AI战略》备忘录,这一转型的核心逻辑在于“速度制胜”,即必须将学习速度武器化,把作战循环时间和技术采用率作为决定性的战场变量\footnote{Secretary of War, \textit{Memorandum: Artificial Intelligence Strategy for the Department of War}, Washington, D.C., January 9, 2026.}。这种对智能化战争形态的追求和对绝对军事优势的执念,直接催生了特朗普政府在数据整合、算力垄断和心理战升级等领域的一系列政策举措。

\section{物理层锁定:AI基础设施的战略动员与排他性构建}

在进攻性技术民族主义的驱动下,特朗普2.0政府在硬实力层面构建了一套严密的物理层锁定机制。这套机制从算力基建、数据要素、采办制度到国际供应链,形成了一个相对封闭的体系,意在从源头控制未来前沿技术的发展路径。

\subsection{垄断科学发现的生产资料与数据基座}

在人工智能时代,算力与高质量数据是定义未来技术权力的核心生产资料。特朗普政府试图通过行政强制力,将这些分散的资源进行国家级整合。

2025年11月,特朗普签署第14363号行政令,正式启动“创世纪任务”(Genesis Mission)。该计划的战略意图清晰而激进:明确要求掌握AI驱动的科学发现能力,即掌握定义未来物理世界的权力。该任务由能源部牵头,建立统一的“美国科学与安全平台”,强制整合联邦政府数十年来积累的科学数据集,并统筹调配包括橡树岭、阿贡在内的国家实验室超级计算资源\footnote{Executive Order 14363, \textit{Launching the Genesis Mission}, Federal Register, Vol. 90, No. 227, November 28, 2025.}。

为配合这一物理层面的垄断,2026年1月,美国战争部常务副部长签署备忘录,宣布将原有的“Advana”数据平台重组为“战争数据平台”。这一举措旨在打造一个能支撑“智能体AI”快速开发与集成的底层数据枢纽,确保从财务管理到杀伤链的每一个环节都运行在统一、受控的数据基座之上\footnote{Deputy Secretary of War, \textit{Memorandum: Transforming Advana to Accelerate Artificial Intelligence and Enhance Auditability}, Washington, D.C., January 9, 2026.}。这种将数据要素国有化、算力资源集中化的做法,实质上是试图垄断科学发现与战争决策的“生产资料”。

\subsection{构建“首席技术官”领导下的敏捷响应机制}

为支撑高强度的科技竞赛,美国战争部在2025年大力推动“规划、计划、预算与执行”(PPBE)改革\footnote{Under Secretary of Defense (Comptroller), \textit{Planning, Programming, Budgeting, and Execution Reform Implementation Report}, January 16, 2025.},并于2026年初完成了国防创新生态的根本性重组。

根据2026年1月发布的《国防创新生态系统转型》备忘录,美国确立了负责研究与工程的副部长作为战争部唯一的“首席技术官”,统摄全局技术方向。同时,将国防创新单元和战略能力办公室升格为“外勤活动”机构,赋予其独立的人事与预算权限\footnote{Secretary of War, \textit{Memorandum: Transforming the Defense Innovation Ecosystem to Accelerate Warfighting Advantage}, Washington, D.C., January 9, 2026.}。

这一体制改革配合了预算机制的创新。新设立的“创新插入增量”预算机制,旨在赋予资金和技术更高的敏捷性。通过设立“障碍移除委员会”快速豁免非法规限制,美军提出了“30天内部署最新AI模型”的采办目标。这一机制折射出特朗普政府“以快制慢”的军事哲学,即通过制度创新将商业技术优势转化为即时的战场优势。

\subsection{“全栈出口”与全球数字势力的物理锁定}

在确保自身技术领先的同时,特朗普政府在国际舞台上推出了具有较强排他性的技术输出战略。

2025年7月,特朗普签署第14320号行政令,正式推出《促进美国AI全栈技术出口计划》。该战略超越了单一产品的销售范畴,转而向全球盟友及摇摆国家推销包含算力中心、数据通道、云服务及底层模型在内的整体解决方案\footnote{Executive Order 14320, \textit{Promoting the Export of the American AI Technology Stack}, Federal Register, Vol. 90, No. 142, July 28, 2025.}。

为配合这一计划,美国动用进出口银行和国际发展金融公司提供融资支持,诱导发展中国家在数字基础设施上全盘采纳美国架构。这种策略的深层逻辑在于实现“物理层锁定”:一旦某国的数字底座由美国标准的硬件和协议构成,其后续的软件生态、数据流转乃至国家安全体系将被迫对美单向透明。这实际上是在全球范围内划定“数字势力范围”,企图将竞争对手从全球关键地缘节点的数字生态中物理剥离。

