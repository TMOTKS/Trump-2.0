\documentclass[12pt,a4paper,fontset=fandol]{ctexart}

% 引入必要的宏包
\usepackage[margin=2.5cm]{geometry} % 设置页边距
\usepackage{setspace} % 设置行距
\usepackage{titlesec} % 标题格式设置
\usepackage[colorlinks=true,linkcolor=black,urlcolor=blue,citecolor=black]{hyperref} % 超链接
\usepackage{indentfirst} % 首行缩进

% ==========================================
% 核心修复:将脚注标号修改为 [1], [2] 格式
% ==========================================
\renewcommand{\thefootnote}{[\arabic{footnote}]}

% 格式设置:符合国内学术期刊规范
\onehalfspacing % 1.5倍行距
\titleformat{\section}{\Large\bfseries\centering}{\thesection}{1em}{} % 一级标题居中
\titleformat{\subsection}{\large\bfseries}{\thesubsection}{1em}{}
\renewcommand{\thesection}{\chinese{section}、}
\renewcommand{\thesubsection}{(\chinese{subsection})}
\pagestyle{plain}

\title{\textbf{科技霸权与认知重塑——特朗普2.0时期“新曼哈顿”战略转型及其影响}}
\author{}
\date{}

\begin{document}

\maketitle

\begin{abstract}
\noindent \textbf{【内容摘要】} 2026年初,随着特朗普第二任期各项政策的密集落地,其执政逻辑已从首个任期单纯的经贸保护主义,深刻进化为以“科技-认知”复合体为核心的霸权护持战略。在《2025国家安全战略》和《2026国家国防战略》的宏观指导下,特朗普政府通过启动“创世纪任务”、实施“战争部全面转型为AI优先作战力量”及重塑国防创新生态,构建了一套集物理控制与认知塑造于一体的战略体系。该体系在硬实力层面,试图通过国家意志强力调配算力与数据资源,建立垄断性的科研与采办体系;在软实力层面,将心理战制度化、常态化,并试图通过“反觉醒”技术标准的输出确立排他性的认知霸权。本文基于2025年至2026年初美国联邦政府发布的关键行政令和政策档案,系统梳理特朗普2.0政府科技与认知新政的实施路径,剖析其背后的权力运作逻辑以及对我国的战略影响和启示建议。

\vspace{1em}
\noindent \textbf{【关键词】} 特朗普2.0;科技霸权;认知战;AI优先战略;大国竞争;数字主权
\end{abstract}

\vspace{2em}

\section*{引言}
\addcontentsline{toc}{section}{引言}

2025年至2026年初,国际政治经济格局经历了冷战后最为剧烈的震荡与重组。这一时期见证了美国大战略的深刻质变。如果说特朗普首个任期的显著特征是粗放式的关税壁垒、退群毁约与实体贸易脱钩,那么其第二个任期则在汲取前期经验的基础上,展现出更为精细、冷酷且具系统性的重塑特征。截至2026年第一季度,随着新版《国家安全战略》的发布、美国国防部极具象征意义地正式更名为“战争部”\footnote{参见2025年底至2026年初美国联邦政府及原国防部发布的一系列备忘录,文件抬头均已变更为“Secretary of War”或“Department of War”。这一机构名称的历史性回归,标志着美国军事战略从防御性威慑向实战化与进攻性的重大转向。},以及一系列关于人工智能、数据主权与技术出口的总统行政令密集落地,美国政府的战略目标已发生根本性转移。华盛顿决策层已不再满足于维持经济增速或制造业回流等指标的表层领先,而是致力于构建一个由美国绝对主导、排斥战略竞争对手的“技术-认知”闭环体系。

在这一新兴的战略体系中,前沿科技特别是生成式人工智能、量子计算与先进半导体,被彻底剥离了其作为人类公共产品的客观属性,被异化为政治权力和地缘博弈的直接延伸;与此同时,认知领域则被美国军方及情报界定义为继陆、海、空、天、网之后的“第六作战域”,是大国博弈必须占领的战略高地\footnote{关于认知域作为“第六作战域”的界定,近年来已在美国军方《联合全域战》(JADO)概念及相关情报界战略评估中被广泛确立,旨在通过信息与心理干预直接影响对手决策逻辑。}。面对这一历史性的新变局,深入剖析特朗普政府在科技动员、认知重塑及规则锁定方面的政策文本,厘清其从物理层面的技术隔离向精神层面的认知霸权升级的战略逻辑,对于准确研判未来国际秩序走向、防范系统性战略风险,以及构建我国的战略应对体系,具有紧迫的现实意义与深远的理论价值。

\section{特朗普2.0“科技-认知”战略转型的动因与逻辑}

特朗普2.0政府之所以在2025年底至2026年初密集且激进地出台极具进攻性的科技与认知政策,并非出于偶然的政策冲动或单纯的选举承诺兑现,而是基于其核心幕僚团队对国际权力格局演变、国内政治极化态势以及智能化战争形态的综合研判。这种被美国战略界内部称为“新曼哈顿”式的战略转型,具有深刻的内生动因与底层逻辑。

\subsection{霸权焦虑与谋求第二次“大分流”的技术民族主义}

随着多极化趋势的深入发展和以金砖国家为代表的“全球南方”群体性崛起,美国战略界对自身综合国力相对衰落的焦虑感空前加剧。特朗普2.0政府的经济与安全智囊们认为,冷战后确立的传统自由市场机制和新自由主义全球化,不仅导致了美国产业空心化,更无法在当前的大国竞争中确保美国在关键技术领域的绝对领先地位。因此,美国必须诉诸极端的技术民族主义甚至国家资本主义手段,以非对称的技术代差来重塑全球权力分配。

这一逻辑在2026年1月美国总统经济顾问委员会发布的重磅报告《人工智能与大分流》中得到了最直白的阐述。该报告明确提出,正如工业革命导致了西方与非西方世界财富与权力的第一次大分流,美国当前必须利用生成式人工智能的爆发期,引发全球经济与军事实力的第二次大分流,以彻底且不可逆地拉开与战略竞争对手的差距\footnote{The Council of Economic Advisers, "Artificial Intelligence and the Great Divergence," Executive Office of the President, January 2026, p. 1.}。

为实现这一宏大目标,美国打破了共和党传统的小政府与自由放任理念,转而利用国家意志强行干预技术生态。报告数据显示,2024年全球企业人工智能投资达到2520亿美元,其中美国私营部门的投资高达940亿美元,远超其他国家\footnote{The Council of Economic Advisers, "Artificial Intelligence and the Great Divergence," Executive Office of the President, January 2026, pp. 9-10.}。然而,特朗普政府认为单纯依靠私营部门仍不足以确立绝对的护城河。在《2026国家国防战略》中,美国战争部直言不讳地提出,必须进行“如同上个世纪世界大战和冷战期间的国家工业动员”\footnote{Secretary of War, "2026 National Defense Strategy," Department of War, January 2026, p. 21.}。这种基于霸权焦虑的国家级动员逻辑,促使美国试图通过整合全社会的算力、资本与数据,建立起对竞争对手的绝对代差优势。其核心目的在于,面对《2026国家国防战略》中重点担忧的多战区同时性问题,通过技术垄断来弥补美军在常规兵力规模和造舰能力上的相对不足,以极化的技术霸权来强行护持其摇摇欲坠的单极霸权体系。

\subsection{右翼保守主义与“身份政治”的外部投射}

特朗普2.0战略的另一显著且极具争议的特征,是将美国国内日益激烈的右翼保守主义与“反觉醒”文化战争,强行延伸至国际战略和全球技术标准领域。在数字时代,算法的伦理标准和底层参数直接决定了信息的筛选、呈现机制以及价值导向。特朗普政府敏锐地意识到,人工智能模型并非价值中立的数学工具,而是蕴含着巨大意识形态能量的政治载体。

《2025国家安全战略》明确将能力与优绩主义列为美国国家复兴的核心原则,并以前所未有的严厉措辞,公开反对多元、公平、包容等被其视为“削弱美国竞争力”、“破坏制度”的激进左翼意识形态\footnote{President of the United States, "National Security Strategy of the United States of America," The White House, November 2025, p. 11.}。这一原本属于美国国内的身份政治诉求,被迅速且强制性地投射到人工智能的军事、政府采购与国家安全应用中。

在2026年1月9日发布的《战争部人工智能战略》备忘录中,美国战争部部长明确提出军事领域发展的指导思想是“摒弃乌托邦式的理想主义,拥抱冷峻的现实主义”。该文件明令禁止美国军方使用任何经过社会意识形态调整的模型,要求模型必须提供客观真实的回应,绝不允许激进意识形态干扰其在杀伤链和情报分析中的应用\footnote{Secretary of War, "Memorandum: Artificial Intelligence Strategy for the Department of War," Washington, D.C., January 9, 2026, p. 5.}。美国总统经济顾问委员会的报告也进一步证实,联邦采购指南已被更新,强制要求政府只能与那些保证其系统免受自上而下的意识形态偏见影响的开发者签订合同\footnote{The Council of Economic Advisers, "Artificial Intelligence and the Great Divergence," Executive Office of the President, January 2026, p. 25.}。这种将技术标准意识形态化的做法,本质上是试图通过行政干预清洗模型中的多元文化参数,确立一套符合美国右翼保守主义价值观的排他性认知标准,进而通过美国的全球技术出口网络,向世界各国输出一种隐蔽的、算法驱动的认知霸权。

\subsection{智能化战争形态演进的现实驱动}

除了宏观的政治与意识形态因素,生成式人工智能与智能体技术的突破性进展,为特朗普政府重塑军事与认知战略提供了最直接的现实物理驱动力。当前,该技术正处于从专用智能向通用人工智能跃迁的临界点。美国战争部深刻认识到,技术赋能的战争模式和武器能力开发,将在未来十年彻底重新定义军事事务的特征与战争的制胜机理\footnote{Secretary of War, "Memorandum: Artificial Intelligence Strategy for the Department of War," Washington, D.C., January 9, 2026, p. 1.}。

在此背景下,特朗普政府确立了将美军全面转型为“人工智能优先”作战力量的宏大目标。这一转型的核心逻辑可以概括为速度制胜。正如战略备忘录所强调的,在可预见的未来,军事较量是一场纯粹的竞速赛,美国必须将学习速度武器化,把作战循环时间和技术采用率作为决定性的战场变量,以超越任何潜在对手的适应与反应能力\footnote{Secretary of War, "Memorandum: Artificial Intelligence Strategy for the Department of War," Washington, D.C., January 9, 2026, p. 4.}。

为了实现这一目标,美国不仅在内部强行打破五角大楼冗长的官僚壁垒,设立障碍移除委员会以实现技术的敏捷插入;更在外部试图通过严苛的出口管制和技术封锁,人为切断竞争对手获取先进算力的途径,从而延缓对手的智能化进程。这种对智能化战争形态的狂热追求和对绝对军事优势的执念,直接催生了特朗普2.0政府在数据整合、算力垄断和心理战升级等领域的一系列激进举措。

\section{物理层锁定:国家意志下的科技动员与垄断体系}

为了实现《人工智能与大分流》报告中规划的战略目标,特朗普2.0政府在硬实力层面彻底抛弃了自由市场的温情脉脉,转而构建了一套严密且极具排他性的物理层锁定机制。这套机制从算力基建、数据要素、采办制度到国际供应链,形成了一个完整的闭环,意在从源头垄断未来前沿技术的发展路径。

\subsection{垄断科学发现的生产资料与数据基座}

在人工智能时代,算力与高质量数据是定义未来物理世界权力的核心生产资料。特朗普政府试图通过行政强制力,将这些分散在联邦各机构的资源进行国家级垄断。

2025年11月,特朗普签署第14363号行政令,正式启动极具野心的“创世纪任务”。该计划的战略意图清晰而激进:明确要求美国必须掌握由前沿技术驱动的科学发现能力。不同于以往分散、竞争性的科研资助模式,“创世纪任务”由美国能源部牵头,建立统一的美国科学与安全平台。该平台强制整合了联邦政府数十年来积累的庞大科学数据集,并统筹调配包括橡树岭、阿贡在内的国家实验室超级计算资源。其核心目标是利用智能体在核聚变、先进半导体材料、生物制药及量子信息科学等六大关键领域实现自动化的科研攻关与技术突破\footnote{Executive Order 14363, "Launching the Genesis Mission," Federal Register, Vol. 90, No. 227, November 28, 2025, pp. 55035-55037.}。

为配合这一国家级物理层面的垄断,美国军方在数据基座的重组上亦采取了激进举措。2026年1月9日,美国战争部常务副部长签署备忘录,宣布将原有的数据平台进行根本性重组,剥离非核心业务,集中打造战争数据平台。这一举措打破了美军各军种间的数据壁垒,旨在构建一个能支撑智能体在全军范围内快速开发与集成的底层数据枢纽,确保从财务审计、后勤调度到前端杀伤链的每一个环节,都运行在统一、受控且高度标准化的数据基座之上\footnote{Deputy Secretary of War, "Memorandum: Transforming Advana to Accelerate Artificial Intelligence and Enhance Auditability," Washington, D.C., January 9, 2026, pp. 1-2.}。这种将数据要素高度国有化、算力资源绝对集中化的做法,实质上是美国政府试图垄断科学发现与战争决策的底层逻辑,从而建立起对他国的绝对代差优势。

\subsection{构建“首席技术官”领导下的敏捷响应机制}

在传统的大国军事竞争中,庞大而僵化的官僚体制往往是阻碍新技术向战斗力转化的最大绊脚石。为支撑高强度的科技竞赛,美国战争部在2025年大力推动规划、计划、预算与执行体制改革\footnote{Under Secretary of Defense (Comptroller), "Planning, Programming, Budgeting, and Execution Reform Implementation Report," January 16, 2025, pp. 1-4.},并于2026年初完成了国防创新生态的根本性重组,试图赋予庞大的战争机器以硅谷初创企业般的敏捷性。

2026年1月9日,战争部发布《国防创新生态系统转型》备忘录,确立了负责研究与工程的副部长作为战争部唯一的首席技术官,统摄全局技术方向。同时,将国防创新单元和战略能力办公室升格为外勤活动机构,赋予其独立的人事与预算权限,旨在彻底消除多头管理的内耗\footnote{Secretary of War, "Memorandum: Transforming the Defense Innovation Ecosystem to Accelerate Warfighting Advantage," Washington, D.C., January 9, 2026, pp. 1-2.}。

这一体制改革配合了极具攻击性的预算机制创新。依托特朗普签署的《大美法案》所提供的巨额资金支持,战争部新设立了创新插入增量预算机制,并成立了由首席技术官领导的障碍移除委员会。该委员会被赋予了极高的特权,能够快速豁免非强制性的法规限制,绕过繁琐的审批程序\footnote{Secretary of War, "Memorandum: Transforming the Defense Innovation Ecosystem to Accelerate Warfighting Advantage," Washington, D.C., January 9, 2026, p. 4.}。这一系列制度设计的终极目标,在《战争部人工智能战略》中被明确量化为:在商业前沿模型公开发布的30天内,完成其在美军作战系统中的部署\footnote{Secretary of War, "Memorandum: Artificial Intelligence Strategy for the Department of War," Washington, D.C., January 9, 2026, p. 4.}。这种机制不仅是为了支撑《2025国家安全战略》中提出的旨在拦截大规模导弹袭击的金色穹顶计划\footnote{President of the United States, "National Security Strategy of the United States of America," The White House, November 2025, p. 3.},更折射出特朗普2.0政府试图通过制度创新,将商业技术优势无缝转化为即时战场优势的战略企图。

\subsection{“全栈出口”与“硅基和平”的数字铁幕}

在确保自身技术研发与军事采办绝对领先的同时,特朗普政府在国际舞台上推出了极具扩张性和排他性的技术输出战略,意图在全球范围内划定数字势力范围。

2025年7月,特朗普签署第14320号行政令,正式推出《促进美国人工智能全栈技术出口计划》。该战略彻底超越了单一软件或硬件产品的销售范畴,转而向全球盟友及摇摆国家推销包含优化硬件、数据通道、云服务及底层大模型在内的整体解决方案\footnote{Executive Order 14320, "Promoting the Export of the American AI Technology Stack," Federal Register, Vol. 90, No. 142, July 28, 2025, pp. 35393-35394.}。

这一出口计划并非单纯的商业行为,而是具有强烈地缘政治色彩的物理层锁定工具。美国总统经济顾问委员会在报告中首次披露了名为“硅基和平”的国际供应链联盟。该联盟将处于供应链上游的设备制造国与下游拥有雄厚资本和能源的数据中心投资国强行捆绑在美国的战车上\footnote{The Council of Economic Advisers, "Artificial Intelligence and the Great Divergence," Executive Office of the President, January 2026, p. 6.}。为配合这一计划,美国动用进出口银行、国际发展金融公司等提供国家级融资支持,诱导“全球南方”等发展中国家在数字基础设施上全盘采纳美国架构。

这种策略的险恶之处在于其深刻的路径依赖与主权剥夺效应。一旦某国的数字底座由美国标准的硬件和闭源协议构成,其后续的软件生态繁衍、数据流转监管乃至国家安全体系的运行,将被迫对美国单向透明。这实际上是美国在21世纪推行的数字新殖民主义,企图通过物理基础设施的排他性,将以中国为代表的竞争对手从全球关键地缘节点的数字生态中物理剥离,强行降下一道割裂世界的数字铁幕。

\section{认知域重塑:心理战的制度化与霸权输出}

进入特朗普2.0时期,美国政府的科技战略已完全超越了硬实力竞争的传统范畴,转而将科技优势特别是大语言模型的话语生成能力,异化为重塑全球认知结构、界定国际规则的底层工具。认知战已被美国从辅助性的战术手段,正式提升为核心的主战样式,力图构建服务于美国优先的全球认知闭环。

\subsection{心理战的战略回归与智能体战术升级}

美国军事战略文化正在经历从隐蔽的信息干预向公开的、进攻性心理操纵的重大转型。《2025国家安全战略》明确指出,过去数十年的国家建设和乌托邦式理想主义侵蚀了美军的战士精神,必须予以全面恢复。

在此宏观指导下,2025年12月2日,美国战争部发布了一份极具冷战色彩的备忘录,正式废弃了沿用多年的、显得相对温和的“军事信息支援行动”称谓,高调恢复使用“心理战”这一术语。该文件毫不掩饰地指出,美国军方将致力于实施旨在影响外国政府、组织、群体及个人的情感、动机、客观推理,并最终改变其行为的计划行动\footnote{Secretary of War, "Memorandum: Changing the Term Military Information Support Operations Back to Psychological Operations," Washington, D.C., December 2, 2025.}。

这一战略回归绝非仅仅是字面意义上的更名,而是在《战争部人工智能战略》中得到了极其恐怖的技术支撑。该战略启动了七大领跑项目,其中“蜂群工场”和“智能体网络”项目,旨在利用智能体直接参与从战役规划到杀伤链执行的决策闭环;而“开放兵工厂”项目则致力于将海量情报在数小时而非数年内转化为武器化的代码与认知攻击素材\footnote{Secretary of War, "Memorandum: Artificial Intelligence Strategy for the Department of War," Washington, D.C., January 9, 2026, pp. 2-3.}。这意味着美军已将操纵对手国民认知、瓦解敌方抵抗意志视为与物理火力打击同等重要的独立作战能力。在大模型的加持下,未来美军将在全球范围内更无顾忌地发动旨在改变对手决策逻辑的认知攻势,认知域的灰度竞争将迅速向高烈度对抗演变。

\subsection{联邦优先权下的“反觉醒”算法标准统一}

在数字时代,大语言模型的伦理标准和对齐机制直接决定了人类社会的信息筛选与历史叙事。特朗普政府敏锐地意识到,美国内部各州不一的监管法律,甚至硅谷科技巨头内部偏向左翼的多元文化主义,正在严重削弱美国对外输出统一认知叙事的能力。

为了消除这种内部噪音,特朗普政府采取了雷厉风行的行政干预手段。2025年1月,特朗普签署第14179号行政令,废除了拜登政府时期强调安全与信任的行政令,扫除了所谓阻碍美国领导力的障碍\footnote{Executive Order 14179, "Removing Barriers to American Leadership in Artificial Intelligence," Federal Register, Vol. 90, No. 20, January 31, 2025, p. 8741.}。随后,在2025年12月签署的第14365号行政令《确保国家政策框架》中,特朗普政府更是动用了联邦优先权,授权美国司法部长成立诉讼特别工作组,专门负责起诉和打压如科罗拉多州等试图立法禁止算法歧视的地方州政府\footnote{Executive Order 14365, "Ensuring a National Policy Framework for Artificial Intelligence," Federal Register, Vol. 90, No. 239, December 16, 2025, pp. 58499-58500.}。

同时,该行政令还动用联邦宽带资金作为要挟,迫使各州放弃对模型输出内容进行多元化约束。结合战争部对采购中“任何合法用途”的强制要求,美国正在通过国家机器强行清洗模型中的觉醒文化参数,建立一套符合右翼保守主义价值观的统一算法标准。当这套经过所谓净化与去意识形态化包装的算法标准,随着全栈出口计划推向全球时,实际上是在向接收国输出一种极具隐蔽性、排他性和同化能力的认知霸权。它将在潜移默化中重塑全球南方国家民众的历史观与价值观。

\subsection{教育体系的军事化重塑与代际认知预置}

认知战的终极防线不仅在于当下的舆论场,更在于未来的人才与思想阵地。特朗普2.0时期将国家安全观念的灌输与技术标准的同化,前推至基础教育阶段,试图从代际源头巩固美国的认知优势。

2025年4月,特朗普签署第14277号行政令《推进美国青少年的教育》。该政策表面上是为了提升青少年的数字技能,但其执行机制却暴露了深刻的国家安全逻辑。该行政令明确提出建立由白宫科技政策办公室牵头,包含国防部、能源部等强力部门参与的白宫教育特别工作组,并举办覆盖全国的总统挑战赛\footnote{Executive Order 14277, "Advancing Artificial Intelligence Education for American Youth," Federal Register, Vol. 90, No. 80, April 28, 2025, pp. 17519-17520.}。

这一政策超越了一般性的基础教育目标,其深层逻辑在于通过联邦政府的直接介入与资金引导,将国家安全需求、美式技术伦理与幼儿园至高中的基础教育深度融合。通过在中小学阶段普及符合国家战略需求即排除了多元包容等理念的课程,美国试图培养一支对美式技术标准有天然亲和力、对国家霸权战略有深刻认同感的原生代后备军。这种从娃娃抓起的战略布局,实际上是在为未来二十年的大国认知博弈进行深度的兵力预置,确保美国在数字化生存的长期竞争中,始终拥有最坚实、最忠诚的人力资源储备。

\section{“新曼哈顿”战略对国际秩序与我国的系统性影响}

特朗普2.0政府通过创世纪任务垄断科研范式、利用大美法案重塑算力基建,以及将心理战制度化等一系列举措,已实质性地构建起一张精密的技术与认知封锁网。这不仅对战后以联合国为核心的多边秩序造成了颠覆性冲击,更在物理与精神双重层面对我国构成了空前严峻的系统性战略挤压。

\subsection{加剧“同时性问题”与“敏捷性突袭”风险}

在传统的地缘政治博弈中,大国的军力部署往往受到地理空间和反应时间的严格物理限制。然而,美国战争部新设立的首席技术官体制及领跑项目机制,正在极大地压缩这种战略缓冲时间。备忘录中明确提出的三十天内部署最新模型及障碍移除委员会的设立,意味着美军正试图具备像软件迭代一样快速更新武器系统和杀伤链的作战能力。

这种对速度制胜的狂热追求,将彻底改变传统的战争准备节奏,并极大地加剧《2026国家国防战略》中重点强调的同时性问题\footnote{Secretary of War, "2026 National Defense Strategy," Department of War, January 2026, p. 13.}。特朗普政府企图通过技术赋能的无人蜂群、智能体网络和自动化指挥系统,在印太、中东、欧洲等多个战区同时保持高强度的非对称介入能力。对于我国而言,这意味着美军在核心利益区域,可能利用技术的敏捷插入,发动难以预警的敏捷性突袭。这种打破传统战略稳定框架的做法,将极大地增加大国间发生战略误判与意外冲突的风险。

\subsection{技术生态阵营化与“数字铁幕”的实质性降临}

特朗普政府推行的全栈技术出口计划与硅基和平联盟,本质上是一种披着高科技外衣的数字新殖民主义。美国总统经济顾问委员会的报告毫不避讳地指出,美国正通过贸易协定和国家资本力量,迫使盟友如欧盟在贸易协定中承诺购买四百亿美元美国芯片,以及中东富国如阿联酋参与投资高达万亿美元的数据中心项目,在数字基础设施上全盘采纳美国架构\footnote{The Council of Economic Advisers, "Artificial Intelligence and the Great Divergence," Executive Office of the President, January 2026, p. 24.}。

这种通过物理层锁定和标准捆绑的做法,正在全球范围内强行降下一道数字铁幕。其深层危害在于:一方面,它企图将广大发展中国家永久固化在产业链的底端,剥夺其数字主权;另一方面,它通过排他性的技术同盟,对我国的科技出海、数字一带一路建设及全球供应链的安全性构成了严重的物理阻断。如果硅基和平联盟得以完全成型,我国在国际数字经济格局中将面临被边缘化和孤立的系统性风险。

\subsection{国际舆论场与规则制定权的系统性压制}

在认知域,美国正将算法标准异化为赤裸裸的霸权工具。通过第14365号行政令等强硬手段,美国在联邦层面强行清除了模型中的多元包容参数,确立了带有强烈右翼保守主义和美国优先色彩的底层算法逻辑。

当这套经过所谓去意识形态化包装、实则深植西方中心主义价值观的模型,借助全栈出口计划推向全球时,其破坏力将远超传统的媒体宣传。大模型作为新一代的信息守门人,将在潜移默化中重塑全球受众尤其是年轻一代的信息获取习惯、历史叙事与客观推理逻辑。这不仅是对我国国际话语权和制度解释权的隐蔽剥夺,更是对全球文明多样性的系统性压制。长此以往,我国在国际多边舞台上的规则制定权和道义感召力将面临被算法静音的危险。

\section{跨越“数字铁幕”:我国的战略应对与路径选择}

面对特朗普2.0政府构筑的科技与认知闭环体系,我国需准确研判其战略意图。在保持战略定力的同时,必须摒弃传统的线性思维,采取灵活务实、跨域协同的应对之策,以真正的多边主义和科技自立自强解构其霸权逻辑。

\subsection{加快国防采办与科研体制的适应性改革}

针对美军学习速度武器化的敏捷突袭风险,我国的国防建设与科研管理体系必须进行深刻的适应性重塑。我国不能再仅以传统的年度预算周期和漫长的装备研制流程来预判对手的行动速度。

建议在中央军委和国防科技工业层面,建立针对美军快速采办、低于门槛重编程和非程序化资金流动的动态监测预警机制。同时,应加快自身的国防采办体制改革,探索设立类似创新快速通道的敏捷响应机制。在确保安全保密的前提下,打通军民融合的堵点,让国内优秀的商业大模型和无人技术能够以月甚至周为单位快速转化为国防实力,确保在技术对抗中不因制度摩擦而丧失宝贵的时间窗口。

\subsection{构建“数字主权”统一战线,反制技术殖民}

美国推行的排他性技术输出,附带着极强的政治附加条件和主权干涉,这恰恰触及了广大全球南方国家的战略痛点。发展中国家虽渴求搭乘技术快车,但绝不希望沦为任何超级大国的数字附庸或数据提供者。

我国应敏锐抓住这一地缘政治契机,全面升级技术出海策略,从单纯的硬件产品输出转向全方位的技术赋能。依托全球发展倡议,我国应高举数字主权和技术不结盟的旗帜,向全球南方国家提供去捆绑、开源可控、独立自主的替代性算力基础设施方案。通过帮助广大发展中国家建立本土化的数据中心、训练符合其本国文化与语言习惯的模型,我国可以构建起广泛而坚实的国际数字统一战线,从根本上对冲和瓦解美国的排他性技术霸权。

\subsection{利用美国联邦体制裂痕实施精准反制}

特朗普2.0时期激进的联邦集权与右翼保守政策,已在美国国内引发了剧烈的政治反弹与社会撕裂。例如,第14365号行政令设立诉讼特别工作组,动用联邦资金要挟并强行压制科罗拉多州、加利福尼亚州等地方政府关于算法公平的立法权\footnote{Executive Order 14365, "Ensuring a National Policy Framework for Artificial Intelligence," Federal Register, Vol. 90, No. 239, December 16, 2025, p. 58499.}。这使得华盛顿联邦政府与自由派重镇及硅谷科技界之间的裂痕空前扩大。

在应对美国的科技打压与认知战攻势时,我国应善于发现并利用其内部的结构性矛盾。在对外战略传播中,应客观揭露其联邦集权对美国自身民主多元价值观的破坏,以及技术霸权对全球数字生态多样性的威胁。同时,我国可积极开展次国家级外交,加强与美国地方州政府、科技界理性力量、高校科研机构及民间社会的务实交流与经贸合作,利用美国联邦体制的缝隙,最大程度地对冲和稀释华盛顿极端反华政策的毒性。

\subsection{强化全民认知防御与复合型人才培养}

面对美国通过第14277号行政令将认知战前推至基础教育领域的长远代际布局\footnote{Executive Order 14277, "Advancing Artificial Intelligence Education for American Youth," Federal Register, Vol. 90, No. 80, April 28, 2025, p. 17519.},我国必须从总体国家安全观的高度,重新审视和升级国民教育与人才培养体系。

在智能化混合战争时代,单纯的理工科硬技能教育已不足以应对复杂的国家安全挑战。我国必须强化全民,特别是青少年的认知免疫力。建议在国民教育体系和国防教育中,系统性地引入算法伦理、信息溯源、深度伪造识别及认知安全教育,提升公众在复杂、高对抗信息环境下的辨别力。同时,国家应出台专项计划,加快培养兼具前沿技术背景、深厚人文底蕴与广阔国际政治视野的复合型战略人才,建立一支既懂底层算法逻辑、又精通大国心理博弈的专业队伍,以防止在未来二十年的代际竞争中出现致命的认知断层。

\section*{结语}
\addcontentsline{toc}{section}{结语}

特朗普2.0政府在科技与认知领域的新政,是一次试图通过强硬的行政手段和国家资本主义,强行逆转历史多极化周期的战略冒险。其核心逻辑在于透支美国现存的技术存量与金融霸权,通过扭曲自由市场规则、重塑科学范式与清洗多元文化,构建一个排他性的科技与认知闭环体系,以期人为制造第二次大分流。

然而,从长远的历史唯物主义视角审视,这种新曼哈顿战略的内在脆弱性与逻辑悖论已然显现。首先,试图通过行政权力强行垄断科学发现的解释权和生产资料,本质上背离了科技创新所绝对依赖的开放、共享与多元精神;长此以往,必将导致美国自身科研生态的僵化与创新活力的枯竭。其次,当反觉醒和去意识形态化成为美国对外输出的新的政治正确时,这种将算法标准异化为霸权工具的做法,不仅难以获得国际社会的广泛认同,反而会极大地刺激全球南方国家对数字主权丧失的深层担忧,进而引发全球范围内的防御性反弹与技术去美化进程。最后,美国国内联邦与地方、保守与自由之间的结构性撕裂,将成为制约其宏大战略目标实现的致命内生阻力。

对于我国而言,面对特朗普2.0政府构筑的数字铁幕与气势汹汹的认知攻势,应始终保持高度的战略定力与制度自信。历史的经验反复证明,单纯的技术封锁难以遏制一个拥有完备工业体系和庞大人才基数的大国崛起;而基于意识形态偏见的认知操纵,也终将被客观的发展事实所无情解构。只要我国坚持科技自立自强,坚定维护国家数字主权,积极践行真正的多边主义,持续拓展平等互利、包容普惠的国际技术合作网络,就一定能有效对冲外部系统性风险,在百年未有之大变局的惊涛骇浪中,牢牢掌握中华民族伟大复兴的战略主动权。

\end{document}

