\documentclass[12pt,a4paper,fontset=fandol]{ctexart}

% 引入必要的宏包
\usepackage[margin=2.5cm]{geometry} % 设置页边距
\usepackage{setspace} % 设置行距
\usepackage{titlesec} % 标题格式设置
\usepackage[colorlinks=true,linkcolor=black,urlcolor=blue,citecolor=black]{hyperref} % 超链接
\usepackage{indentfirst} % 首行缩进

% ==========================================
% 核心修复:将脚注标号修改为 [1], [2] 格式
% ==========================================
\renewcommand{\thefootnote}{[\arabic{footnote}]}

% 格式设置:符合国内学术期刊规范
\onehalfspacing % 1.5倍行距
\titleformat{\section}{\Large\bfseries\centering}{\thesection}{1em}{} % 一级标题居中
\titleformat{\subsection}{\large\bfseries}{\thesubsection}{1em}{}
\renewcommand{\thesection}{\chinese{section}、}
\renewcommand{\thesubsection}{(\chinese{subsection})}

% ==========================================
% 核心修复:去除页眉中的一级标题,保持页面干净
% ==========================================
\pagestyle{plain}

\title{\textbf{科技霸权与认知重塑——特朗普2.0时期对华人工智能战略的逻辑与应对}}
\author{}
\date{}

\begin{document}

\maketitle

\begin{abstract}
\noindent \textbf{【内容摘要】} 人工智能已成为大国战略竞争的核心场域。2026年初,随着特朗普第二任期各项政策的密集落地,美国对华科技战略已从选择性脱钩升级为以“进攻性技术民族主义”为内核的系统性遏制。在这一战略框架下,美国试图通过“星际之门”计划和严苛的出口管制,在物理层实现算力基础设施的排他性构建;同时,利用智能体驱动的认知对抗与算法价值观投射,在认知层谋求话语霸权。然而,该战略的封闭性与零和博弈内核,与科技创新的开放性相冲突,在实践中面临着开源生态冲击、能源瓶颈及盟友战略自主诉求等多重内在悖论与实施困境。面对美国构筑的“数字铁幕”,中国应坚持高水平科技自立自强,构筑开源创新生态,携手“全球南方”践行真正的多边主义,统筹发展与安全,构建国家认知安全与社会韧性体系,以期在全球人工智能治理体系变革中掌握战略主动。

\vspace{1em}
\noindent \textbf{【关键词】} 特朗普2.0;人工智能;进攻性技术民族主义;认知战;大国竞争;数字主权
\end{abstract}

\vspace{2em}

\section*{引言}
\addcontentsline{toc}{section}{引言}

当前,世界正经历百年未有之大变局,大国战略竞争与第四次科技革命相互交织。人工智能技术因其高度的战略性、基础性与军民两用性,已超越单纯的技术属性,成为重塑全球经济、军事与权力结构的关键变量\footnote{鲁传颖、才悦:《特朗普2.0时期中美人工智能博弈的新阶段》,《太平洋学报》2025年第10期,第15页。}。在此背景下,科技创新深度嵌入地缘政治博弈,人工智能安全逐渐成为国家安全体系的核心议题。

2025年至2026年初,随着特朗普再次当选美国总统,美国大战略发生深刻质变。如果说特朗普首个任期的显著特征是粗放式的关税壁垒与实体贸易脱钩,那么其第二个任期则在汲取前期经验的基础上,展现出更为精细且具系统性的重塑特征。特朗普政府将人工智能定位为维系全球科技霸权、控制产业链以及遏制战略竞争对手的关键场域。华盛顿决策层不再满足于维持经济指标的表层领先,而是致力于构建一个由美国主导、排斥战略竞争对手的“技术-认知”闭环体系。

在这一新兴的战略体系中,前沿科技特别是生成式人工智能,被异化为政治权力和地缘博弈的延伸;与此同时,认知领域则被美国军方及情报界定义为继陆、海、空、天、网之后的“第六作战域”,是大国博弈着力争夺的战略高地\footnote{韩娜、邹初妤:《智能认知对抗:理论演化、对抗机制与安全风险》,《国际安全研究》2025年第6期,第49页。}。面对这一历史性的新变局,深入剖析特朗普政府在人工智能基础设施布局与智能认知对抗方面的政策文本,厘清其从物理层面的技术隔离向精神层面的认知霸权升级的战略逻辑,对于准确研判未来国际秩序走向、防范系统性战略风险,以及构建我国的战略应对体系,具有重要的现实意义与理论价值。

\section{特朗普2.0时期人工智能战略的深层动因与逻辑}

特朗普2.0政府在2025年底至2026年初密集出台极具进攻性的科技与认知政策,并非出于偶然的政策冲动,而是基于其核心幕僚团队对国际权力格局演变、国内政治极化态势以及智能化战争形态的综合研判。这种战略转型具有深刻的内生动因与底层逻辑。

\subsection{进攻性技术民族主义与霸权护持}

随着多极化趋势的深入发展,美国战略界对自身综合国力相对衰落的焦虑感明显上升。特朗普2.0政府的经济与安全智囊认为,传统的自由市场机制无法在当前的大国竞争中确保美国在关键技术领域的绝对领先地位。因此,美国政策取向明显向“进攻性技术民族主义”倾斜,试图以非对称的技术代差来重塑全球权力分配。

进攻性技术民族主义是技术政治化进程中的一种理论形态与实践范式,它将技术实力视为国家生存、安全与霸权维系的基础,认为国家最大安全在于垄断能够颠覆现状的“决定性技术优势”\footnote{郭畅、冯晓青:《进攻性技术民族主义与特朗普2.0时期的美国对华人工智能战略》,《太平洋学报》2025年第11期,第59-60页。}。这一逻辑在2026年1月美国总统经济顾问委员会发布的报告《人工智能与大分流》中得到了直白的阐述。该报告提出,美国当前必须利用生成式人工智能的爆发期,引发全球经济与军事实力的第二次“大分流”,以不可逆地拉开与战略竞争对手的差距\footnote{The Council of Economic Advisers, "Artificial Intelligence and the Great Divergence," Executive Office of the President, January 2026, p. 1.}。

为实现这一目标,美国打破了传统的小政府与自由放任理念,转而利用国家意志强行干预技术生态。在《2026国家国防战略》中,美国战争部提出必须进行“如同上个世纪世界大战和冷战期间的国家工业动员”\footnote{Secretary of War, "2026 National Defense Strategy," Department of War, January 2026, p. 21.}。这种基于霸权焦虑的国家级动员逻辑,促使美国试图通过整合全社会的算力、资本与数据,建立起对竞争对手的代差优势,以极化的技术霸权来护持其单极霸权体系。

\subsection{“科技-工业复合体”的利益驱动与“内政外化”}

特朗普2.0战略的另一显著特征,是国内政治生态对外交与科技政策的深刻形塑。近年来,美国国内政治愈发成为影响其外交政策的关键变量,“内政驱动外交”的态势日益凸显。特朗普政府通过将国内治理困境归咎于外部威胁,呈现出明显的“内政外化”治理逻辑\footnote{孙涵:《特朗普“美国优先2.0”外交战略及其内政驱动因素》,《国际论坛》2025年第5期,第25页。}。

在这一逻辑下,硅谷科技精英与右翼政治力量的合流发挥了关键作用。部分科技精英推崇“技术加速主义”,认为技术的自驱进步将促进社会发展,主张探索能够在释放技术红利与赚取商业利益间保持平衡的路径\footnote{金灿荣、申欣钰:《特朗普政府人工智能政策的国家安全逻辑》,《太平洋学报》2025年第8期,第45页。}。特朗普与这部分科技精英达成共识,形成了一种“以权力换利润、以利润换权力”的相互支持关系\footnote{马贺非、毛维准:《特朗普政府人工智能基础设施政策探析》,《现代国际关系》2025年第4期,第13-14页。}。科技巨头通过提供私人资本换取在政策制定中的话语权,而政府则通过支持科技公司的创新发展来兑现其政治承诺。这种“科技-工业复合体”的深度内嵌,不仅推动了美国人工智能政策向重创新、轻监管的方向倾斜,也使得技术议题不可避免地被意识形态化和安全化,成为转移国内矛盾、凝聚选民基础的政治工具。

\subsection{智能化战争形态演进与本体安全焦虑}

生成式人工智能与智能体技术的突破性进展,为特朗普政府重塑军事与认知战略提供了现实驱动力。当前,人工智能技术正处于从专用智能向通用人工智能跃迁的临界点。美国战争部深刻认识到,技术赋能的战争模式和武器能力开发,将在未来十年重新定义军事事务的特征与战争的制胜机理\footnote{Secretary of War, "Memorandum: Artificial Intelligence Strategy for the Department of War," Washington, D.C., January 9, 2026, p. 1.}。

与此同时,美国面临着传统物质安全与国家“本体安全”的双重压力。本体安全指国家内部免受社会分裂、认同销蚀、政府合法性削弱等威胁的状态与能力\footnote{金灿荣、申欣钰:《特朗普政府人工智能政策的国家安全逻辑》,《太平洋学报》2025年第8期,第44页。}。为了应对这些挑战,特朗普政府确立了将美军全面转型为“人工智能优先”作战力量的宏大目标。这一转型的核心逻辑在于“速度制胜”,即必须将学习速度武器化,把作战循环时间和技术采用率作为决定性的战场变量,以超越潜在对手的适应与反应能力\footnote{Secretary of War, "Memorandum: Artificial Intelligence Strategy for the Department of War," Washington, D.C., January 9, 2026, p. 4.}。这种对智能化战争形态的追求和对绝对军事优势的执念,直接催生了特朗普政府在数据整合、算力垄断和心理战升级等领域的一系列政策举措。
